% !TEX program = xelatex
% !TEX encoding = UTF-8
\documentclass[a4paper]{book}

% % Greek font
% \usepackage[T2A,T1]{fontenc}
% \usepackage[utf8]{inputenc}
\usepackage[greek,english]{babel}

\usepackage{ctex}
\punctstyle{kaiming}
\setCJKmainfont[BoldFont=FZHei-B01, ItalicFont=FZKai-Z03
% FZNewKai-Z03
]{FZShuSong-Z01}
\setCJKfamilyfont{fangzhengkaiti}{FZKai-Z03}
\setmonofont{Consolas}

% % % uncomment these lines to enable Times New Roman font
% \usepackage{newtxtext}
% \usepackage{newtxmath}

\usepackage{multicol}
% Page Margin
\usepackage[% inner = 1in, outer = 1.5in % for my default book
    inner=1in,
    outer=1.5in,
    bottom=1.5in,
]{geometry}

% Pronunciation
\usepackage{tipa}
\newcommand\yb[1]{\mbox{\upshape{\textipa{[#1]}}}}
\usepackage{ifxetex}
\ifxetex
  \usepackage{substitutefont}
  \substitutefont{T3}{\rmdefault}{cmr}
\fi
% https://tex.stackexchange.com/questions/464082/xelatex-font-shape-t3-lmr-m-n-undefinedfont-using-t3-cmr-m-n-insteadfont/464087
\usepackage{lmodern}
% https://tex.stackexchange.com/questions/58087/how-to-remove-the-warnings-font-shape-ot1-cmss-m-n-in-size-4-not-available
\usepackage{anyfontsize} % Font size will probably be problem

\usepackage{xcolor}
    \definecolor{gl}{RGB}{0,68,124} % 钢蓝, 短语/例句的颜色
    \definecolor{zs}{RGB}{174,13,22} % 棕色, 扩展的颜色

\usepackage{tcolorbox}
    \tcbuselibrary{breakable, skins} %使用它的 breakable
\usepackage{multicol}%分栏
    \setlength{\columnsep}{1cm}
\usepackage{imakeidx}
    \makeindex[title=单词表, options=-s zimu.ist]
    \makeindex[name=prefix, title=前缀, options=-s zimu.ist]
    \makeindex[name=suffix, title=后缀, options=-s zimu.ist]
    \makeindex[name=root, title=词根, options=-s zimu.ist]
    \makeindex[name=tubiao, title=补充内容]
    \indexsetup{noclearpage}
    \usepackage{idxlayout}
    \idxlayout{columns=4, rule=0.6pt} %索引分4栏
    \usepackage[hidelinks]{hyperref}

% % suffix and perfix
% \newtcolorbox{qianzhuihezi}{%定义前缀盒子
%     colback=black!10,boxrule=0mm,colframe=black!10,
%     width=\linewidth, %宽度
%     boxsep=0mm, %文字和框的间距
%     left=3mm, %文字和左边框的间距
%     after skip=2mm,
%     before skip=5mm}
% \newcommand\zhui[2]{
%     \begin{qianzhuihezi}
%     \textbf{#1 #2}
%     \end{qianzhuihezi}}
%
% \newcommand\qz[2]{%前缀
%     \zhui{#1-\ \index[prefix]{#1}}{#2} \label{#1}}%
% \newcommand\qzz[3]{%俩前缀
%     \zhui{#1- = #2-\ \index[prefix]{#1}\index[prefix]{#2}}{#3} \label{#1} \label{#2}}%
% \newcommand\qzzz[4]{%仨前缀
%     \zhui{#1- = #2- = #3-\
%     \index[prefix]{#1}\index[prefix]{#2}\index[prefix]{#3}}{#4}
%     \label{#1}\label{#2}\label{#3}
%     }
% \newcommand\hz[2]{%后缀
%     \zhui{-#1\ \index[suffix]{#1}}{#2} \label{#1}}%
% \newcommand\hzz[3]{%俩后缀
%     \zhui{-#1 = -#2\ \index[suffix]{#1}\index[suffix]{#2}}{#3} \label{#1} \label{#2}}%
% \newcommand\hzzz[4]{%仨后缀
%     \zhui{-#1 = -#2 = -#3\
%     \index[suffix]{#1}\index[suffix]{#2}\index[suffix]{#3}}{#4}
%     \label{#1}\label{#2}\label{#3}}
% \renewcommand\r[2]{%词根
%     \zhui{#1\ \index[root]{#1}}{#2} \label{#1}}%
% \newcommand\rr[3]{%俩词根
%     \zhui{#1 = #2\ \index[root]{#1}\index[root]{#2}}{#3} \label{#1} \label{#2}}%
% \newcommand\rrr[4]{%仨词根
%     \zhui{#1 = #2 = #3\
%     \index[root]{#1}\index[root]{#2}\index[root]{#3}}{#4}
%     \label{#1}\label{#2}\label{#3}}

\renewcommand\c[1]{%单词
    \par\noindent
    \normalsize\normalfont
    \textbf{#1}\ \index{#1}\small}
\newcommand\cc[1]{
\c{#1}
}
% \newcommand\cc[1]{%二级单词
%     \par \setlength\parindent{5pt}%
%     \hangindent 20pt%
%     \normalsize\normalfont
%     \textbf{#1}\ \index{#1}\small}
\newcommand\dy[1]{\ \ {\color{gl}\slshape{#1}}}%短语

\usepackage{enumitem}
%\usepackage{amsmath} %input black square
\usepackage{setspace}
\newcommand\tb[2]{%图表盒子
%    \setlength{\leftmargin}{0mm} %左边界
%    \setlength{\parsep}{0ex} %段落间距
%    \setlength{\topsep}{1ex} %列表到上下文的垂直距离
%    \setlength{\itemsep}{0mm} %条目间距
%    \setlength{\labelsep}{0.3em} %标号和列表项之间的距离,默认0.5em
%    \setlength{\itemindent}{0.2em} %标签缩进量
%    \setlength{\listparindent}{0em} %段落缩进量
    \vskip 2mm
    \begin{spacing}{1.2}
    \noindent \small
    \begin{tcolorbox}
      [colframe=white, coltext=zs, coltitle=zs,
      size=fbox,
      opacityback=0, opacitybacktitle=0, opacityframe=0,
      sharp corners=all,
      breakable, enhanced jigsaw,
      title={\rule{0.7em}{0.7em} \textbf{#1}}]
      \begin{enumerate}[leftmargin=*,itemsep=-2pt,topsep=0mm]
          %\setlength{\itemsep}{0pt}
          \CJKfamily{fzfs} #2
      \end{enumerate}
    \end{tcolorbox}
    \end{spacing}
    \index[tubiao]{#1}
    }
\newcommand\tbw[2]{ %图表,里面是文字,不是列表
    \begin{spacing}{1.2} %段落中的行距
    {\color{zs} %\enlargethispage{1mm}
    \vskip 2mm \hrule \vskip 2mm
    \noindent \small
    \rule{0.7em}{0.7em} \textbf{#1}\index[tubiao]{#1}%标题
    % \textsl
    {\CJKfamily{fzfs}
        \\ #2}
    \vskip 2mm \hrule \vskip 3mm
    \normalfont \normalsize}
    \end{spacing}
    }

%fc-list -f "%{family}\n" :lang=zh > zhfont.txt
\setCJKfamilyfont{youyuan}{FZFangSong-Z02}%引入幼圆字体 YouYuan
\setCJKfamilyfont{fzfs}{FZFangSong-Z02} % 方正仿宋
\newcommand\jiakuang[1]{\CJKfamily{youyuan}\setlength{\fboxsep}{0.3mm}\fbox{\tiny#1\normalsize}\normalfont
}%加框
\newcommand\fk[1]{\jiakuang{#1}}%调用加框, 这一行实际可省略

\newcommand\fangkuangXiaozi[2]{
    % \begin{spacing}{1.0}
        % \vskip 1pt
        \par\noindent
        % \small
        \footnotesize
        \jiakuang{#1} \CJKfamily{fzfs} #2
        % \normalfont\normalsize
        % \vskip 2pt
    % \end{spacing}
}
\newcommand\zy[1]{\fangkuangXiaozi{注意}{#1}}
\newcommand\jf[1]{\fangkuangXiaozi{记法}{#1}}
% \newcommand\zt[1]{\fangkuangXiaozi{真题}{#1}}
% \newcommand\fwb[1]{\fangkuangXiaozi{笔记者注}{#1}}
\newcommand\fy[1]{\fangkuangXiaozi{翻译}{\normalfont\emph{#1}}}

%词性的格式
\newcommand\pos[1]{ {\itshape#1}.}
\renewcommand {\a}   {\pos{adj}}
\newcommand   {\ad}  {\pos{adv}}
\newcommand   {\n}   {\pos{n}}
\renewcommand {\v}   {\pos{v}}
\newcommand   {\vi}  {\pos{vi}}
\newcommand   {\vt}  {\pos{vt}}
\newcommand   {\vn}  {\pos{vn}}
\newcommand   {\nv}  {\pos{nv}}
\renewcommand {\int} {\pos{int}}
\newcommand   {\prep}{\pos{prep}}
\newcommand   {\pl}  {\pos{pl}}

% \tolerance=1 %下面几行是防止换行
% %\emergencystretch=\maxdimen
% \hyphenpenalty=10000
% \hbadness=10000
\usepackage[none]{hyphenat}

\renewcommand\sl{$\cdots$ }%省略
\newcommand\ty{$\sim$ }%同义词,后面不接任何东西,所以要有空格
\newcommand\tyc{$\sim$}%同义词,由于是词语,后面接 s、ed 等

% \usepackage{fancyhdr}
% \pagestyle{fancy}
% % \lhead{}
% % \chead{}
% % bfseries
% \chead{\fangsong\small 笔记排版 / 范文斌 (178-6567-1650)}
% % \lfoot{From: K. Grant}
% \cfoot{}
% \fancyfoot[LE,RO]{\thepage}
% \fancyfoot[LO,RE]{\fangsong\small 仅供参考学习,切勿挪作他用,版权归属新东方}
% \renewcommand{\headrulewidth}{0pt}

\usepackage{draftwatermark}
\SetWatermarkText{\textbf{仅供参考}}
\SetWatermarkColor[gray]{0.98}

\newcounter{timu}%题目
\setcounter{timu}{1}
\newcommand\UnivYear{测试·年份}
\newcommand\setUY[1]{\renewcommand\UnivYear{#1}}
\newcommand\ti[1]{
    \normalfont\normalsize
    \begin{tcolorbox}[%
        colback=black!3!white, colframe=black!3!white,
        size=title,
        boxrule=1pt, before skip=8pt, after skip=5pt,
        breakable,
        enhanced jigsaw]
    \textsf{\textbf{\arabic{timu}.}} (\fangsong\UnivYear\normalfont) #1
    \end{tcolorbox}\stepcounter{timu}
}
\newcommand\hh{\underline{\hbox to 10mm{}} }

% new page with each section
\usepackage{titlesec}
\newcommand{\sectionbreak}{\clearpage}

\usepackage{ifthen}
\usepackage{array}
\newcommand\abcdsty[1]{#1.} % style of ABCD label
\newcommand{\fourch}[4]{%~\hfill(\qquad)\\
\begin{tabular}{*{4}{@{}p{0.25\textwidth}}}\abcdsty{A}~#1 & \abcdsty{B}~#2 & \abcdsty{C}~#3 & \abcdsty{D}~#4\end{tabular}}
\newcommand{\twoch}[4]{%~\hfill(\qquad)\\
\begin{tabular}{*{2}{@{}p{0.5\textwidth}}}\abcdsty{A}~#1 & \abcdsty{B}~#2\end{tabular}\\\begin{tabular}{*{2}{@{}p{0.5\textwidth}}}\abcdsty{C}~#3 & \abcdsty{D}~#4\end{tabular}}
\newcommand{\onech}[4]{%~\hfill(\qquad)\\
\abcdsty{A}~#1 \\ \abcdsty{B}~#2 \\ \abcdsty{C}~#3 \\ \abcdsty{D}~#4}

\newlength\widthcha
\newlength\widthchb
\newlength\widthchc
\newlength\widthchd
\newlength\widthch
\newlength\tabmaxwidth
\newlength\fourthtabwidth
\newlength\halftabwidth

\newcommand{\choice}[4]{\\
    \setlength\tabmaxwidth{1\textwidth}
    \setlength\fourthtabwidth{0.25\textwidth}
    \setlength\halftabwidth{0.5\textwidth}
    \settowidth\widthcha{AM.#1}
    \setlength{\widthch}{\widthcha}
    \settowidth\widthchb{BM.#2}
    \ifthenelse{\widthch<\widthchb}{\setlength{\widthch}{\widthchb}}{}
    \settowidth\widthchb{CM.#3}
    \ifthenelse{\widthch<\widthchb}{\setlength{\widthch}{\widthchb}}{}
    \settowidth\widthchb{DM.#4}
    \ifthenelse{\widthch<\widthchb}{\setlength{\widthch}{\widthchb}}{}
    \ifthenelse{\widthch<\fourthtabwidth}{\fourch{#1}{#2}{#3}{#4}}
    {\ifthenelse{\widthch<\halftabwidth\and\widthch>\fourthtabwidth}
    {\twoch{#1}{#2}{#3}{#4}}
    {\onech{#1}{#2}{#3}{#4}}}
}

% path of include or input
\makeatletter
\providecommand*\input@path{}
\newcommand\addinputpath[1]{
\expandafter\def\expandafter\input@path
\expandafter{\input@path{#1}}}
\addinputpath{section/}
\makeatother

\begin{document}
% \begin{titlepage}
% \hfill
% {\centering\huge\heiti
% 新东方托福核心词汇
% \normalfont\normalsize
% \\
% \kaiti\large
% 主讲 /
% \\
% 笔记排版 / 范文斌 (178-6567-1650)}
% \vspace{3cm}\normalfont\normalsize
% \end{titlepage}
% \tableofcontents

\chapter{真题}
\section{中国人民大学 2001 年}
\setUY{人大2001}
\begin{multicols}{2}
\ti{And the topic of ``fat'' is forbidden. Even the slightest paunch betrays that one is losing
the trim and \hh of youth.
\choice{vague}{vigor}{vogue}{vulgar}}
\c{forbid} \yb{f@"bId} \vt 禁止;阻止;阻碍;妨碍
\cc{ban} \yb{b\ae n} \vt 禁止,下令禁止;剥夺权利 \n 禁止,禁令;谴责;诅咒,诅骂
\cc{prohibit} \yb{pr@"hIbIt} \vt 禁止,阻止,防止;不准许
\c{betray} \yb{bI"treI} \vt 对…不忠;背叛;出卖;泄露
\c{trim} \yb{trIm} \vt 修剪;整理;装饰 \a 整齐的,整洁的;修长的;苗条的 \n 整齐;修剪;健康状态;装束 \vi 削减
\c{youth} \yb{ju:T} \n 年轻;青年,小伙子;初期,少年(青年)时期;〔集合词〕青少年们
\tb{表示``走''的词根词缀}{
\item \cg{vag}-,来自 way
\item -\cg{gress},向前
\item -\cg{vade}
\item -\cg{cess}
\item -\cg{ven} = -\cg{vent},来自 went,go 的过去式
}
\c{vague} \yb{veIg} \a 模糊的;(思想上)不清楚的;(表达或感知)含糊的;暧昧的
\cc{vaguely} \yb{"veIgli} \ad 含糊地;茫然地;暧昧的
\c{progress} \yb{"pr@Ugres} \n 进步;前进;〔生物学〕进化;(向更高方向)增长 \v (使)进步,(使)进行;发展;促进 \vi 发展;(向更高方向)增进
\c{aggress} \yb{@"gres} \v 攻击,侵犯
\jf{a- 表示加强}
\cc{aggressive} \yb{@"gresIv} \a 侵略的,侵犯的,攻势的;〔美〕有进取心的,积极行动的;有进取心的,有闯劲的;好争斗的,借故生端的,爱打架的,要打架的
\zy{由上下文判断 aggressive 的含义}
\c{invade} \yb{In"veId} \v 侵入,侵略;进行侵略;蜂拥而入,挤满;(疾病,声音等)袭来,侵袭 \vt 涌入;侵袭;侵犯;干扰
\cc{invasion} \yb{In"veIZn} \n 入侵,侵略;侵害,侵犯;侵袭
\cc{evade} \yb{I"veId} \vt 逃避,躲避;避开;规避;逃脱 \vi 逃避;规避;逃脱
\jf{\cg{e}- 出去,\cg{ex}- = \cg{ec}- = \cg{es}- = \cg{epi}- 向外}
\jf{epi- 可表示上、中、下}
\cc{inevitable} \yb{In"evIt@bl} \a 不可避免的;必然发生的;〔非正〕总会发生的;必然发生的事
\zy{元音 a、e、i、o、u 的作用是非常不重要的,仅用于辅助发音。如 evade 替换元音变为 inevitable 中的 -evit-}
\zy{辅音替换,类比通假字,如 t-d、p-b、k-g、f-v}
\c{process} \yb{"pr@Uses} \n 过程;工序;做事方法;工艺流程 \yb{pr@U"ses} \vt 加工;处理;审阅;审核 \vi 列队行进
\c{proceed} \yb{pr@"si:d} \vi 进行;前进;(沿特定路线)行进;(尤指打断后)继续说 \n 收入,获利
\zy{听不出来单词的原因是发音不对}
\c{excess} \yb{Ik"ses} \n 超过;超额量;多余量;放肆 \yb{"ekses} \a 超重的,过量的,额外的
\c{exceed} \yb{Ik"si:d} \vt 超过;超越;胜过;越过…的界限 \vi 突出,领先
\zy{80\% 的单词会认,20\% 会写,如翻译、写作中}
\c{avenue} \yb{"\ae v@nju:} \n 林荫路;大街;途径,手段;〔美〕(南北向)街道
\c{venture} \yb{"ventS@} \n 冒险事业;冒险;冒险行动;商业冒险 \vt 冒…的危险;拿…冒险;用…进行投机;以…做赌注 \vi 冒险前进,冒险行事;猜测 \dy{\ty at}
\c{adventure} \yb{@d"ventS@} \n 冒险活动;冒险经历;奇遇
\c{vigor} \yb{"vIg@} \n 活力;智力;(语言等的)气势;(动,植物的)强健
\jf{\cg{vi}- 表示看、生命,生命的意思来自法语 vie}
\c{vision} \yb{"vIZn} \n 视力,视觉;想象;幻象;想象力
\c{view} \yb{vju:} \n 看法;风景;视域 \vt 看;看待
\c{witness} \yb{"wItn@s} \n 目击者,见证人;〔法〕证人;证据 \vt 出席或知道;作记录;提供或作为…的证据;(年份)见证 \vi 做证人;见证
\c{evident} \yb{"evId@nt} \a 明显的,明白的;昭著
\cc{evidence} \yb{"evId@ns} \n 证词;证据;迹象;明显 \vt 使明显;显示;表明;证实
\c{vivid} \yb{"vIvId} \a 生动的;(记忆、描述等)清晰的;(人的想像)丰富的;(光、颜色等)鲜艳的,耀眼的
\c{virus} \yb{"vaIr@s} \n 病毒;病毒性疾病;毒素,毒害;〔计算机科学〕计算机病毒
\c{vital} \yb{"vaItl} \a 维持生命所必需的;至关重要的;生死攸关的;生机勃勃的
\zy{可在写作中替换 important}
\cc{fatal} \yb{"feItl} \a 致命的,攸关的;毁灭性的,严重的;命中注定的;重大的
\c{fate} \yb{feIt} \n 命运;天意;命中注定的事(尤指坏事)
\c{vibrate} \yb{vaI"breIt} \v (使)振动,颤动;摆动;犹豫;激动
\zy{-\cg{ate} 动词结尾}
\cc{vibration} \yb{vaI"breISn} \n 摆动;震动;感受;(偏离平衡位置的)一次性往复振动
\cc{mode} \yb{m@Ud} \n 方式;状况;时尚,风尚;调式;模式
\c{vogue} \yb{v@Ug} \n 时尚,流行;时髦的事物 \a 流行的,时髦的
\c{folk} \yb{f@Uk} \n 民族;人们;〔口〕家属,亲戚;民间音乐 \a 民间的;普通平民的;流传民间的;普通百姓的 \dy{Volks Wagon 大众汽车}
\zy{volk、folk 相互替换}
\cc{volk} \yb{fO:lk} \n 民族人民

\ti{All specialists agree that the most important consideration with diet drugs is carefully
\hh the risks benefits.
\choice{valuing}{evaluating}{estimating}{weighing}}
\c{benefit} \yb{"benIfIt} \n 利益,好处;救济金,津贴;义演,义卖;恩惠,恩泽 \vt 有益于,有助于;使受益;得益,受益
\jf{\cg{bene}- 好}
\cc{profit} \yb{"pr6fIt} \n 收益,得益;利润;红利;净值利润率 \vt 有益于…,对…有益的;得益;创利润 \vi 有益;获利
\cc{benediction} \yb{""benI"dIkSn} \n 祝福;(礼拜结束时的)赐福祈祷;恩赐;(大写)(罗马天主教)祈求上帝赐福的仪式
\jf{\cg{dict}- 说,措辞、用语}
\c{value} \yb{"v\ae lju:} \n 价值,价格;意义,涵义;重要性;(邮票的)面值 \vt 评价;重视,看重;估价,给…定价
\cc{estimate} \yb{"estIm@t} \n 估计,预测;报价,预算书;评价,判断 \vt 估计,估算;评价,评论;估量,估价 \dy{\ty the value}
\cc{valuable} \yb{"v\ae lju@bl} \a 贵重的,宝贵的;有价值的;可评估的 \n 贵重物品,财宝
\cc{invaluable} \yb{In"v\ae lju@bl} \a 非常宝贵的;无法估计的;无价的;金不换
\cc{unvalued} \yb{"2n"v\ae lju:d} \a 不受重视的;无价值的;不足道的;未评价过的
\cc{evaluate} \yb{I"v\ae ljueIt} \vt 评价;求…的值(或数);对…评价;〔数学、逻辑学〕求…的数值 \vi 评价,估价
\jf{\cg{e}- 外,-\cg{val}- 说,说出来}
\c{equivalent} \yb{I"kwIv@l@nt} \a 相等的,相当的,等效的;等价的,等积的;〔化学〕当量的 \n 对等物;〔化学〕当量
\jf{\cg{equ}- 相等,-\cg{val}- 价值,等价的}
\cc{equal} \yb{"i:kw@l} \a 相等的,平等的;平稳的;势均力敌的;胜任的 \vt 等于;比得上;使相等;同样看待 \n 同样的人;相等的数量;能与之比拟的东西;(地位、实力等)相同的人 \vi 相等,相平(与out连用) \dy{be \ty to}
\cc{equation} \yb{I"kweIZn} \n 方程式;等式;相等;〔化学〕 反应式
\cc{equator} \yb{I"kweIt@} \n 赤道;(平分球形物体的面的)圆;(任何)大圆
\c{valid} \yb{"v\ae lId} \a 有效的;有法律效力的;正当的;健全的
\jf{有效 = 有价值}
\cc{invalid} \yb{In"v\ae lId} \a 无效的;不能成立的;有病的;病人用的 \vt 使伤残;使退役;失去健康 \n 病人,病号;残废者;伤病军人 \vi 变得病弱;因病而奉命退役
\cc{available} \yb{@"veIl@bl} \a 可获得的;有空的;可购得的;能找到的
\c{weigh} \yb{weI} \vt 称…的重量;权衡,考虑;用手掂估;使吃重 \vi 具有重要性;重量为;起锚;成为…的重荷 \n 权衡;称重量 \dy{\ty A against B 权衡A和B}
\c{weight} \yb{weIt} \n 重量,体重;重担,重任;重要;〔统〕权,加重值,权重 \vt 加重于,使变重;使负重,使负担或压迫;〔统〕使加权,附加权重值于

\ti{Chinese often shake my hand and don't let go. They talk away contentedly, \hh of my
discomfort and struggle to disengage my hand.
\choice{oblivious}{patent}{obvious}{pernicious}}
\c{contentedly} \yb{k@n"tentIdlI} \ad 满足地,安心地
\c{struggle} \yb{"str2gl} \vi 搏斗;奋斗;努力;争取 \n 打斗;竞争;奋斗
\c{oblivious} \yb{@"blIvi@s} \a 忘却的;健忘的;不注意的;不知道的
\jf{\cg{ob}- 加强或否定,这里是否定,-liv- 是 live}
\c{patent} \yb{"p\ae tnt , "peItnt} \n 专利;专利品;专利权;专利证 \a 专利的;显然,显露;明摆着的 \vt 获得…专利,给予…专利权;取得专利权 \dy{\ty emphasis that something bad is obvious. }
\c{obvious} \yb{"6bvi@s} \a 明显的;显著的;平淡无奇的;自明
\cc{evident} \yb{"evId@nt} \a 明显的,明白的;昭著;彰
\c{pernicious} \yb{p@"nIS@s} \a 很有害的,恶性的,致命的,险恶的;引起巨大伤害的;毁灭性的
\cc{noxious} \yb{"n6kS@s} \a 有害的,有毒的

\ti{The word “foolish” is too mild to describe your behavior, I would prefer the word \hh.
\choice{ideological}{idyllic}{idiotic}{idiomatic}}
\zy{填一个更加 foolish 的词}
\c{foolish} \yb{"fu:lIS} \a 愚蠢的;鲁莽的;荒谬的,可笑的;混
\c{ideological} \yb{""aIdI@"l6dZIkl} \a 思想的;意识形态的
\cc{ideology} \yb{""aIdi"6l@dZi} \n 思想(体系),思想意识;意识形态;观念学;空想,空论
\c{idyllic} \yb{I"dIlIk} \a 田园诗般的;牧歌的;质朴宜人的;平和欢畅的 \dy{\ty is a short poem that describe peaceful life. }
\c{idiotic} \yb{""Idi"6tIk} \a 白痴的;憨痴;呆头呆脑;憨头憨脑
\c{idiomatic} \yb{""Idi@"m\ae tIk} \a 符合语言习惯的,成语的;含有习语的 \ad 符合语言习惯地 \n 符合语言习惯
\cc{idiom} \yb{"Idi@m} \n 习语,成语;方言,土语;(语言)风格;惯用语法 \dy{\ty means that a group of words whose meaning is different from individual word. }

\ti{Because of its excellence in quality, for the last two years, Audi car has \hh Germany's Touring Car Championship.
\choice{conquered}{contested}{dominated}{determined}}
\c{tour} \yb{tU@} \n 旅行,观光;巡回演出;任职期;轮班 \vt 到…旅游;在…巡回演出 \vi 观光;巡回
\c{excellence} \yb{"eks@l@ns} \n 优秀,卓越;优点;美德
\c{conquer} \yb{"k6Nk@} \vt 征服;攻克;打败(敌人);克服 \vi 得胜,胜利 \dy{I came, I saw, I \tyc ed. }
\c{contest} \yb{"k6ntest} \vt 竞争,为…而奋争;辩驳 \vi 争斗;竞争;争夺 \n 比赛;竞赛;搏斗
\jf{\cg{con}- 一起,一起考试即竞争}
\c{dominate} \yb{"d6mIneIt} \v 支配,影响;占有优势;在…中具有最重要的特色;在…中拥有最重要的位置
\cc{dome} \yb{d@Um} \n 圆屋顶;像圆屋顶一样的东西;圆顶体育场 \vt 加圆屋顶于…上 \vi 成圆顶状
\jf{\cg{dome}- 家,来自 home}
\cc{domestic} \yb{d@"mestIk} \a 家庭的,家的;国内的;驯养的;热心家务的 \n 佣人;国货 \dy{gross \ty product 国内生产总值}
\cc{domesticate} \yb{d@"mestIkeIt} \vt 驯养;使爱家;适应家庭生活;引进
\cc{domicile} \yb{"d6mIsaIl} \n 住处;永久住处 \vt 定居
\c{determine} \yb{dI"t3:mIn} \v (使)下决心,(使)做出决定 \vt 决定,确定;判定,判决;使决定;限定 \vi 〔法〕了结,终止,结束
\cc{determination} \yb{dI""t3:mI"neISn} \n 决心;决定,确定;〔物〕测定,计算

\ti{What we consider a luxury at one time frequently becomes a \hh, many families find
that ownership of two cars is indispensable.
\choice{fashion}{necessity}{proclivity}{nuisance}}
\zy{填 luxury 的反义词}
\c{luxury} \yb{"l2kS@ri} \n 奢侈,豪华;奢侈品,美食,美衣;乐趣,享受;不常有的乐趣(或享受、优势) \a 奢华的,豪华的
\c{proclivity} \yb{pr@"klIv@ti} \n 倾向,癖性
\c{nuisance} \yb{"nju:sns} \n 讨厌的东西(人,行为)麻烦事;非法妨害,损害;麻烦事

\ti{The chief editor thought he took some liberties with the original in translation. So it
was necessary that he make the \hh suggested.
\choice{alterations}{alternatives}{alternations}{altercations}}
\c{liberty} \yb{"lIb@ti} \n 自由;许可权;放肆,无礼;解放,释放
\jf{来自 labor}
\cc{labor} \yb{"leIb@} \n 劳动;努力;工作;劳工 \v 努力争取(for);苦干;详细分析 \a 劳工的,工会的;(常大写)(英国或英联邦国家)工党的 \vi (指引擎)缓慢而困难地运转
\cc{collaborative} \yb{k@"l\ae b@r@tIv} \a 合作的;协作的
\jf{\cg{col}- = \cg{con}- = \cg{com}- 一起,-labor- 劳动,即合作}
\cc{cooperative} \yb{k@U"6p@r@tIv} \a 合作的;协助的;共同的 \n 合作社,联合体
\cc{operate} \yb{"6p@reIt} \v 运转;操作;经营;管理 \vi 开刀;(对…)动手术;动手术;(在某地)采取军事行动 \vt 操作,控制,使运行
\cc{operation} \yb{""6p@"reISn} \n 操作,经营;手术
\cc{liberal} \yb{"lIb@r@l} \n 自由主义者;自由党党员 \a 开明的;自由的;慷慨的;不拘泥的
\c{original} \yb{@"rIdZ@nl} \a 原始的;独创的;最初的;新颖的 \n 原文;原型;原件;怪人
\cc{orient} \yb{"O:rient} \vt 标定方向;使…向东方;以…为参照;使熟悉或适应 \vi 转向东方;使调整或者成为一条直线 \n 东方;亚洲,亚洲各国
\cc{oriental} \yb{""O:ri"entl} \a (抽象的)东方的;东方人的;东方文化的;优质的 \n 东方人;东亚人
\cc{origin} \yb{"6rIdZIn} \n 起源,根源;出身;〔数〕原点,起点;〔解〕(筋,神经的)起端
\cc{originate} \yb{@"rIdZIneIt} \vt 引起;创始,创作;开始,发生;发明 \vi 起源于,来自;产生;起航
\c{suggest} \yb{s@"dZest} \vt 建议,提议;暗示;使想起;启示
\c{alteration} \yb{""O:lt@"reISn} \n 变化,改变;变更
\cc{alter} \yb{"O:lt@} \vt 改变;更改;改建(房屋);(人)变老 \vi 改变;修改
\c{alternative} \yb{O:l"t3:n@tIv} \a 替代的;备选的;其他的;另类的 \n 可供选择的事物;二选一的
\cc{alternate} \yb{O:l"t3:n@t} \a 轮流的;交替的;间隔的;代替的 \vi 交替;轮流 \vt 使交替;使轮流 \n 〔美〕(委员)代理人;候补者;替换物 \dy{That was a week of \ty rain and sunshine. 那是晴雨交替的一周。}
\c{altercation} \yb{""O:lt@"keISn} \n 争辩,争吵
\cc{altercate} \yb{"O:lt@keIt} \v 争论,口角

\ti{Many well-educated people don't believe that \hh will endanger freedom of speech.
\choice{cencership}{censureship}{sensorship}{censorship}}
\c{censorship} \yb{"sens@SIp} \n 审查制度;审查机构;审察员的职权;〔心〕潜意识中的抑制力
\c{censor} \yb{"sens@} \n 监察官,检查员 \vt 审查,检查;审查(书刊等);检查(信件等);删改 \dy{A person whose job is to examine the parts of the movie or the book that considered to be very offensive or immoral. }
\cc{offensive} \yb{@"fensIv} \a 无礼的,冒犯的,唐突的;讨厌的,令人不快的;进攻(性)的,攻击的,攻势的 \n 进攻,攻势
\cc{immoral} \yb{I"m6r@l} \a 不道德的;邪恶的;猥亵的;悖德
\cc{censorious} \yb{sen"sO:ri@s} \a 苛评的,吹毛求疵的
\cc{censure} \yb{"senS@} \n 指责,谴责;责备;斥责 \vt 指责;谴责;责备;批评 \vi 谴责,责备 \dy{strong criticism | \ty sb. for sth. }

\ti{The \hh of “snake” is simply this: a legless reptile with a long, thin body.
\choice{connotation}{denomination}{donation}{denotation}}
\zy{这里填``描述''意思的词,共同部分是 -note-}
\c{note} \yb{n@Ut} \n 笔记;便笺;音符;钞票 \vt 注意;记录;对…加注释;指出
\cc{notable} \yb{"n@Ut@bl} \a 值得注意的;显著的;著名的 \n 名人;显要人物
\cc{outstanding} \yb{aUt"st\ae ndIN} \a 杰出的;显著的;凸出的;未完成的
\cc{exceptional} \yb{Ik"sepS@nl} \a 优越的;杰出的;例外的;独特的 \n 超常的学生
\cc{excellent} \yb{"eks@l@nt} \a 卓越的;杰出的;优秀的;太好了
\cc{extraordinary} \yb{Ik"strO:dnri} \a 非凡的;特别的;非常奇特的
\cc{extra} \yb{"ekstr@} \a 额外的,补充的,附加的;特大的,特别的 \n 附加物,额外的事物;临时演员;号外;上等产品,上品 \ad 额外地;格外地,特别地 \dy{Extra 益达口香糖 | extra large (XL) 超大号}
\cc{ordinary} \yb{"O:dnri} \a 普通的;平常的;一般的;平庸的
\cc{brilliant} \yb{"brIli@nt} \a 明亮的;〔非正式〕美好的;闪耀的;才华横溢的 \n 宝石;钻石
\cc{remarkable} \yb{rI"mA:k@bl} \a 异常的,引人注目的,;卓越的;显著的;非凡的,非常(好)的
\jf{\cg{re}- 反复,-mark- 标记}
\c{distinguish} \yb{dI"stINgwIS} \vi 区分,辨别,分清;辨别是非 \vt 区分,辨别,分清;辨别出,识别;引人注目,有别于;使杰出,使著名
\cc{distinguished} \yb{dI"stINgwISt} \a 卓越的;著名的;受人尊敬的;显得重要的
\c{denotation} \yb{""di:n@U"teISn} \n 指称意义
\cc{denote} \yb{dI"n@Ut} \vt 指代;预示;代表;意思是
\c{connotation} \yb{""k6n@"teISn} \n 内涵,含义;言外之意
\c{denomination} \yb{dI""n6mI"neISn} \n 宗派;教派;(钱的)面额
\jf{\cg{nom}- 是 name}
\c{donation} \yb{d@U"neISn} \n 捐赠,赠送;捐款;捐赠物
\jf{\cg{do}n- 给予}
\cc{Pandora} \yb{p\ae n"dO:r@} \n 潘多拉 \dy{\tyc 's box 潘多拉魔盒}
\jf{\cg{pan}- 全部,-do- = \cg{don}- 给予}
\cc{donate} \yb{d@U"neIt} \v (尤指向慈善机构)捐赠;献(血);捐(血);捐献(器官) \n 捐赠;捐献
\cc{condone} \yb{k@n"d@Un} \vt 容忍,宽恕,原谅 \dy{If someone slaps you on one cheek, turn to them the other also. If someone takes your coat, do not withhold your shirt from them. (Luke 6:29) 有人打你这边的脸,连那边的脸也由他打。有人夺你的外衣,连里衣也由他拿去。(《圣经》路 6:29)}
\cc{dote} \yb{d@Ut} \vi 溺爱,宠爱,过分地喜爱
\c{snake} \yb{sneIk} \n 蛇;奸险的人;卑劣的人;蛇形浮动汇率制 \vt 沿…曲折前进(或延伸) \vi 曲折前进(或延伸);蛇行,悄悄地爬行
\jf{据说蛇曾经受到上帝的惩罚,剥夺了皮毛和四肢,只剩光溜溜的一条,名字便是由 naked 的演化而来}
\cc{naked} \yb{"neIkId} \a 裸体的,裸露的;率直的,赤裸裸的;缺乏保护;不加掩饰的

\ti{When the opposing player fouled John, he let his anger \hh his good sense and hit the body back.
\choice{got the feel of}{got the hang of}{got the better of}{got the worst of}}
\zy{选项的意思分别是习惯、熟悉、占上风、占下风}
\c{player} \yb{"pleI@} \n 演员;〔体〕运动员;〔音〕演奏者;播放机
\c{oppose} \yb{@"p@Uz} \v 反对,抗争;使相对;使对照;抵制
\zy{\cg{op}- = \cg{ob}- 加强或反对}
\cc{pose} \yb{p@Uz} \v 使摆姿势;以…身份出现;招摇;炫耀 \vt 提出;造成(威胁、问题等);引起;产生 \n 姿势;姿态;装腔作势;伪装
\cc{opponent} \yb{@"p@Un@nt} \n 对手;反对者;敌手 \a 对立的;敌对的
\jf{-\cg{ent} 表示人}
\c{foul} \yb{faUl} \a 〔体〕违反规则的,犯规的;邪恶的;难闻的,有恶臭的;下流的 \v 纠缠,纠结;〔体〕违反规则的 \vt 弄脏,使污秽;使纠缠;使阻塞 \vi 腐烂;缠结 \n 犯规;缠结;碰撞 \ad 不正当地,犯规地;碰撞地;争执不和地

\ti{Although this book claims to be a biography of George Washington, many of the
incident are \underline{imaginary}.
\choice{fascinating}{factitious}{fastidious}{fictitious}}
\c{imaginary} \yb{I"m\ae dZIn@ri} \a 想像中的,假想的,虚构的;〔数〕虚数的;幻;虚幻
\c{fascinating} \yb{"f\ae sIneItIN} \a 迷人的,有极大吸引力的;使人神魂颠倒的 \v 使…陶醉(fascinate的ing形式)
\cc{fascinate} \yb{"f\ae sIneIt} \vt 使着迷;使神魂颠倒;蛊惑 \vi 入迷
\cc{fascist} \yb{"f\ae SIst} \n 法西斯主义者;法西斯分子 \a 法西斯主义的;法西斯主义者的;法西斯党的
\jf{法西斯主义是束缚人的思想}
\c{factitious} \yb{f\ae k"tIS@s} \a 不自然的;虚假的;人为的;人造的
\jf{\cg{fac}- = \cg{fec}- = \cg{fic}- 做,如 fact、factory 做事情的地方}
\zy{cat 猫,c- 是手的象形,-t 表示钩,插入元音 -a- 使得单词可发音}
\cc{fiction} \yb{"fIkSn} \n 小说,虚构的文学作品;虚构的或想像出的事,并非完全真实的事;编造,虚构
\jf{fiction,非常神,小说}
\c{fastidious} \yb{f\ae "stIdi@s} \a 挑剔的;讲究的;苛求的;(微生物等)需要复杂营养的
\cc{fuss} \yb{f2s} \n 忙乱;大惊小怪;大惊小怪的人;争吵 \vi 忙乱;大惊小怪;抱怨 \vt 使烦恼,使烦忧;使急躁 \v 瞎忙一气,过分关心
\jf{fuss 发丝,对象身上的头发}
\c{fictitious} \yb{fIk"tIS@s} \a 虚构的,编造的;假定的,虚设的;小说式的;假装的
\c{claim} \yb{kleIm} \vt 声称;断言;需要;索取 \vi 提出要求 \n (根据权利而提出的)要求;声称;断言;索赔 \dy{baggage \ty 行李提取处}
\zy{claim 缩写自 call,第一、三个字母,比如中文里``红色警戒''简称``红警'',没有人叫它``色戒''}
\cc{proclaim} \yb{pr@"kleIm} \vt 宣告,公布;表明;赞扬,称颂
\zy{\cg{pro}- 向前,来自希腊神仙 Prometheus 普罗米修斯的名字}
\cc{exclaim} \yb{Ik"skleIm} \v 呼喊;惊叫;大声说
\cc{acclaim} \yb{@"kleIm} \n 赞扬;欢呼;欢迎 \v 拥护;喝彩;称誉
\c{reclaim} \yb{rI"kleIm} \vt 开拓,开垦;感化;取回;沙化;回收再利用;改造某人,使某人悔改 \vi 抗议,喊叫 \n 改造,感化;再生胶
\jf{\cg{re}- 反复、反对}
\cc{declaim} \yb{dI"kleIm} \v (像演讲般)说话 \vi 抨击;(在公共场合)大声抗议
\jf{\cg{de}- 否定、加强语气,``掷地有声''即 declaim}
\c{biography} \yb{baI"6gr@fi} \n 传记;档案;传记体文学;个人简介
\jf{\cg{bio}- 生命,-\cg{graph} 写,记录生命即传记}
\cc{geography} \yb{dZi"6gr@fi} \n 地理(学);地形,地势;布局
\jf{\cg{geo}- 土地,-\cg{graph}- 写,写一些土地的东西即地理学,geo- 来自希腊神话 Gaea 地母盖亚}
\cc{geology} \yb{dZi"6l@dZi} \n 地质学;(某地区的)地质情况;地质学的著作
\c{accident} \yb{"\ae ksId@nt} \n 意外事件;事故;机遇,偶然;附属品
\cc{incident} \yb{"InsId@nt} \n 事件,事变;小插曲;敌对行动;骚乱 \a 〔法〕附带的;〔光〕入射的;易有的,附随的
\jf{\cg{cid}- = \cg{cad}- = \cg{cas}- = \cg{cat}- 下落}
\c{casual} \yb{"k\ae Zu@l} \a 偶然的;临时的;随便的;非正式的 \n 临时工人;〔军〕待命士兵;没有固定工作的劳动者;不定期领取救济金的人;便装;便鞋;临时工
\c{catastrophe} \yb{k@"t\ae str@fi} \n 大灾难;惨败;悲剧的结局;地表突然而猛烈的变动,灾变
\jf{-\cg{astro} = -\cg{aster} 星星,即 star 字母的重组,AstroBoy 阿童木}
\cc{astrology} \yb{@"str6l@dZi} \n 占星术;占星学;原始天文学
\cc{astronomy} \yb{@"str6n@mi} \n 天文学
\cc{astronomer} \yb{@"str6n@m@} \n 天文学者,天文学家
\cc{astronaut} \yb{"\ae str@nO:t} \n 宇航员;太空人;航天员
\cc{disaster} \yb{dI"zA:st@} \n 灾难;彻底的失败;不幸;祸患

\ti{The trade of designed to \underline{facilitate} further cooperation between Chinese auto
industries and overseas auto industries.
\choice{promote}{protect}{preserve}{prolong}}
\c{facilitate} \yb{f@"sIlIteIt} \vt 促进,助长;使容易;帮助
\c{promote} \yb{pr@"m@Ut} \vt 促进,推进;提升,助长;促销;使(学生)升级;宣传,推广 \vi 成为王后或其他大于卒的子
\jf{\cg{mo}- 移动,move 移动、感动}
\cc{remote} \yb{rI"m@Ut} \a (时间上)遥远的;远离的;远程的;微小的 \n 远程操作;遥控器 \vt 使…向远方延伸,把…延展到远处 \dy{\ty area}
\jf{\cg{re}- 相反,-\cg{mo}- 移动,朝着相反的方向移动即遥远的}
\cc{remove} \yb{rI"mu:v} \vt 开除;去除;脱掉,拿下;迁移 \vi 迁移,移居;离开 \n 距离,差距;移动
\cc{removal} \yb{rI"mu:vl} \n 免职;除去;移走;搬迁
\cc{emotion} \yb{I"m@USn} \n 情感,感情;情绪;感动,激动
\cc{emotional} \yb{I"m@US@nl} \a 表现强烈情感的;令人动情的;易动感情的;感情脆弱的
\c{promotion} \yb{pr@"m@USn} \n 促进,增进;提升,升级;(商品等的)推广,促销;发扬
\cc{prompt} \yb{pr6mpt} \a 敏捷的;迅速的;立刻的;准时的 \v 促使;导致;鼓励;提示 \n 激励;提示符;提词;提示 \ad 准时地 \dy{take \ty action 采取及时的行动}
\cc{prominent} \yb{"pr6mIn@nt} \a 突出的,杰出的;突起的;著名的
\c{preserve} \yb{prI"z3:v} \vt 保护;保持,保存;腌制食物;防腐处理 \vi 保鲜;保持原状;做蜜饯;禁猎 \n 蜜饯;防护用品;禁猎地;独占的事物(或范围) \dy{\tyc ed egg = hundred-year egg 皮蛋}
\jf{-serve 来自 save 保存、拯救}
\jf{\cg{pro}- = \cg{pre}- = \cg{pri}- = \cg{pur}- 前}
\c{conserve} \yb{k@n"s3:v} \vt 保护,保藏,保存;[化,物]使守恒;将…做成蜜饯 \n 果酱,蜜饯 \dy{\ty natural resources 保护自然资源}
\cc{nature} \yb{"neItS@} \n 自然;天性;天理;类型
\cc{conservative} \yb{k@n"s3:v@tIv} \n 保守的人;(英国)保守党党员,保守党支持者 \a 保守的;(英国)保守党的;(式样等)不时新的 \dy{Most of them are rather \tyc. 他们大多数都相当保守。}
\zy{would rather 宁愿,rather than 而不是}
\c{deserve} \yb{dI"z3:v} \vt 应受;应得;值得 \vi 应受报答;应得报酬;应得赔偿;应受惩罚 \dy{Congratulations! You \ty it. 恭喜!这是你应得的。 | You \tyc! 活该!}
\c{reserve} \yb{rI"z3:v} \n 储备;保护区;保留;替补队员 \vt 储备;保留;预约 \vi 预订 \a 保留的;预备的 \dy{\ty a taxi}
\cc{reservation} \yb{""rez@"veISn} \n 保留;预订,预约;保留地,专用地 \dy{make a \ty for seats}
\c{prolong} \yb{pr@"l6N} \vt 延长,拉长;拖延,延期
\c{facilitate} \yb{f@"sIlIteIt} \vt 促进,助长;使容易;帮助 \dy{\ty means ``to make the process more easier''. }
\cc{facility} \yb{f@"sIl@ti} \n 设备,设施,便利条件;容易;能力,天赋;灵巧 \dy{water-supply \ty}
% 2021-01-30 12:34:08 Wenbin, FAN @ SHU
\cc{faculty} \yb{"f\ae klti} \n 能力,才能;全体教职员;〔英〕(大学的)专科,系;特权,特许
\jf{cult- 来自 culture}

\ti{He was concerned only with \underline{mundane} matters, especially the daily stock market quotations.
\choice{rational}{obscure}{worldly}{eminent}}
\zy{题眼在 daily}
\c{daily} \yb{"deIli} \a 每日的,日常的;一日的;每日一次的;每个工作日的 \n 日报;(不寄宿的)仆人,白天做家务的女佣 \ad 每日;逐日;每周日;日复一日地
\c{mundane} \yb{m2n"deIn} \a 平凡的;宇宙的;寻常的;世俗的
\c{concern} \yb{k@n"s3:n} \vt 涉及,关系到;使关心,使担忧;参与 \n 关心;关系,有关;顾虑;公司或企业
\c{rational} \yb{"r\ae Sn@l} \a 神智清楚的;理性的;理智的;合理的 \n 合理的事物;〔数〕有理数;懂道理的人,人类;〔英〕合理的服装
\jf{ration- 来自 reason}
\cc{irrational} \yb{I"r\ae S@nl} \a 不合理的,荒谬的;无理性的;〔数〕无理的 \n 无理数;无理性的生物
\cc{ratiocination} \yb{""r\ae ti""6sI"neISn} \n 〔正〕推论,推理
\c{obscure} \yb{@b"skjU@} \a 昏暗的,朦胧的;晦涩的,不清楚的;隐蔽的;不著名的,无名的 \vt 使…模糊不清,掩盖;隐藏;使难理解 \n 某种模糊的或不清楚的东西
\jf{\cg{ob}- 加强、否定,这里是否定,-scure 看成 sky,``不好的天空'' 即阴暗、模糊、晦涩的}
\c{worldly} \yb{"w3:ldli} \a 世间的,世上的;尘世的;追逐名利的,鄙俗的 \ad 世俗地;世故地
\c{eminent} \yb{"emIn@nt} \a (指人)知名的,杰出的,卓越的;(指品质、特性)明显的,显著的,突出的;显赫的;闻达
\jf{联想 prominent 记忆}

\ti{The earthquake that occurred in India this year was a major \underline{calamity} in which a great many lives were lost.
\choice{casualty}{catastrophe}{catalogue}{crusade}}
\c{lost} \yb{l6st} \a 失去的;迷路的;不知所措的 \v 遗失,失去( lose的过去式和过去分词);(使)失去(所需要的东西,尤指钱);(因事故、年老、死亡等)损失;浪费
\cc{loss} \yb{l6s} \n 损失,减少;丢失,遗失;损耗,亏损;失败 \dy{economic \ty 经济损失}
\cc{economy} \yb{I"k6n@mi} \n 节约;经济;理财;秩序
\cc{economics} \yb{""i:k@"n6mIks} \n 经济学;经济,国家的经济状况
\cc{economic} \yb{""i:k@"n6mIk} \a 经济的;经济学的;合算的;有经济效益的
\cc{economical} \yb{""i:k@"n6mIkl} \a 节约的;经济的;合算的
\c{calamity} \yb{k@"l\ae m@ti} \n 灾祸,灾难;不幸之事;困苦;不幸
\c{casualty} \yb{"k\ae Zu@lti} \n 伤亡(人数);事故,横祸;受害者,死伤者;损坏
\c{casual} \yb{"k\ae Zu@l} \a 偶然的;临时的;随便的;非正式的 \n 临时工人;〔军〕待命士兵;没有固定工作的劳动者;不定期领取救济金的人;便装;便鞋;临时工
\c{catalogue} \yb{"k\ae t@l6g} \n 目录,一览表;〔美〕大学情况便览;展览目录;产品样本 \vt 为…编目录;登记分类;记载,列入目录;登记(某人、某事的)详情
\jf{\cg{cat}- 向下,-\cg{log}- 说,如 -ology- 学说中有个 -log- 说,接着往下说即目录}
\jf{\cg{log}- = \cg{loc}- = \cg{loq}- 说}
\cc{loquacious} \yb{l@"kweIS@s} \a 爱说话的,多嘴的;嘴碎;贫嘴
\jf{-cious 表示比较极端的形容词}
\c{crusade} \yb{kru:"seId} \n 〔史〕十字军东征;(宗教性的)圣战;改革运动 \v 加入十字军;从事改革运动;讨伐
\jf{crus- 即 cross}
\cc{cross} \yb{kr6s} \n 十字架;十字形饰物;杂交品种;痛苦 \vi 交错而行;横渡;越境 \vt 杂交;横跨,穿越;划掉;使相交 \a 坏脾气的, 易怒的;相反的,反向的
\cc{crucial} \yb{"kru:Sl} \a 关键性的,极其显要的;决定性的;十字形的
\zy{可替换 important}
\cc{crisis} \yb{"kraIsIs} \n 危机;危难时刻;决定性时刻,紧要关头;转折点 \dy{ecomonic \ty | environmental \ty}

\ti{The doctors were worried because the patient did not \underline{recuperate} as rapidly as they had expected.
\choice{withdraw}{emerge}{recover}{uncover}}
\c{expect} \yb{Ik"spekt} \vt 期望;预料;要求;认为(某事)会发生 \vi 预期;期待;怀胎;怀孕
\jf{-\cg{pect} = -\cg{spect} 看,\cg{e}- = \cg{ex}- = \cg{ec}- = \cg{es}- 向外}
\cc{prospect} \yb{"pr6spekt} \n 前景;期望;眺望处;景象 \vi 勘探;勘察;(矿等)有希望;有前途 \vt 找矿;对…进行仔细调查
\cc{prosperous} \yb{"pr6sp@r@s} \a 繁荣的,兴旺的;富裕的;幸福的,运气好的;良好的
\cc{prosperity} \yb{pr6"sper@ti} \n 繁荣;兴旺,昌盛;成功
\cc{prospectus} \yb{pr@"spekt@s} \n 内容说明书;(即将出版的书等的)内容介绍,简介;计划书,意见书;(讲义等的)大纲
\c{suspect} \yb{ s@"spekt} \vt 猜疑(是);怀疑,不信任;怀疑…有罪 \n 嫌疑犯 \vi 怀疑 \a 可疑的
\cc{suspicious} \yb{s@"spIS@s} \a 可疑的;猜疑的,怀疑的;多疑的,不信任的;多心
\c{recover} \yb{rI"k2v@} \vt 恢复;重新获得;找回;〔正〕恢复(适当的状态或位置) \vi 恢复健康(体力、能力等)
\c{recuperate} \yb{rI"ku:p@reIt} \vi 恢复,复原;弥补 \vt 使恢复;〔化〕同流换热
\cc{restore} \yb{rI"stO:} \vt 归还;交还;使恢复;修复 \v 恢复(某种情况或感受);使复原;使复位;使复职
\c{withdraw} \yb{wID"drO:} \vt 撤走;拿走;撤退;(从银行) 取 (钱) \vi 撤退;(从活动或组织中) 退出
\c{emerge} \yb{i"m3:dZ} \vi 出现,浮现;暴露;摆脱
\cc{urge} \yb{3:dZ} \vt 催促;推进,驱策;力劝,规劝;极力主张 \n 刺激,冲动;推动力 \vi 催促;强烈要求,竭力主张
\cc{urgent} \yb{"3:dZ@nt} \a 急迫的;催促的;强求的;极力主张的
\cc{merge} \yb{m3:dZ} \v (使)混合;相融;融入;渐渐消失在某物中
\jf{没 mò}
\cc{submerge} \yb{s@b"m3:dZ} \v 淹没;把…浸入;沉没,下潜;使沉浸 \vt 淹没;把…浸入;沉浸 \vi 淹没;潜入水中;湮没
\cc{emergency} \yb{i"m3:dZ@nsi} \n 紧急情况;突发事件;非常时刻 \a 紧急的,应急的
\cc{ambulance} \yb{"\ae mbj@l@ns} \n 救护车;野战医院
\cc{surge} \yb{s3:dZ} \n 汹涌;激增;大量;奔涌向前 \v 汹涌;使强烈地感到;激增;飞涨
\c{uncover} \yb{2n"k2v@} \vt 揭开…的盖子;揭露,发现;拿下(头上)戴的东西,脱(帽);将(狐)赶出 \vi 发现,揭示

\ti{The purchaser of this lorry is protected by the manufacturer's \underline{warranty} that he will replace any defective part for five years of 50,000 miles.
\choice{prohibition}{insurance}{prophecy}{guarantee}}
\c{purchase} \yb{"p3:tS@s} \v 购买;采购;换得;依靠机械力移动 \n 购买;购买行为;购置物;紧握
\c{manufacture} \yb{""m\ae nju"f\ae ktS@} \vt 制造,生产;捏造,虚构;加工;从事制造 \n 大量制造;批量生产;工业品
\jf{\cg{pan}- = \cg{ban}- = \cg{man}- = \cg{fan}- 手,其中 pan- 也表示全部,\cg{fac}- = \cg{fec}- = \cg{fic}- 做}
\c{manufacturer} \yb{""m\ae nju"f\ae ktS@r@} \n 制造商,制造厂;厂主;〔经〕厂商
\c{warranty} \yb{"w6r@nti} \n 保证,担保;〔法〕(商品等的)保单;根据,理由;授权,批准
\cc{warrant} \yb{"w6r@nt} \n 授权证;许可证;正当理由;依据 \vt 保证,担保;授权,批准;辩解
\jf{warr- 看成 wall,``一夫当关,万夫莫开''的感觉}
\c{defective} \yb{dI"fektIv} \a 有错误的;有缺陷的,有瑕疵的;〔语〕变化不全的;智力低于正常的 \n 身心有缺陷的人;变化不全的词
\c{prohibition} \yb{""pr@UI"bISn} \n 禁令,禁律;〔美〕禁酒,〔美史〕禁酒时期;〔法〕诉讼中止令
\cc{prohibit} \yb{pr@"hIbIt} \vt 禁止,阻止,防止;不准许
\cc{forbid} \yb{f@"bId} \vt 禁止;阻止;阻碍;妨碍
\cc{ban} \yb{b\ae n} \vt 禁止,下令禁止;剥夺权利;〔古〕诅咒 \n 禁止,禁令;谴责;诅咒,诅骂;革出教门
\jf{用手(ban-)挡住}
\cc{forbidding} \yb{f@"bIdIN} \a 冷峻的,令人生畏的 \v 禁止( forbid的现在分词 );妨碍;阻碍;阻止
\cc{banner} \yb{"b\ae n@} \n 横幅;旗,旗帜;标语;大字标题 \a 第一流的,第一位的;杰出的;领先的,为首的;突出地支持(某一政党)的 \dy{hold high the great \ty 高举伟大旗帜}
\c{insurance} \yb{In"SU@r@ns} \n 保险,保险业;保险费;预防措施 \a 〔体〕巩固球队领先局面,使对手不能因增加一分而成平局的 \dy{medical \ty | health \ty}
\cc{insure} \yb{In"SU@} \vt 保证;确保;为…保险;投保 \vi 买卖或卖保险
\jf{\cg{in}- 加强}
\cc{assure} \yb{@"SU@} \vt 向…保证;使…确信;〔英〕给…保险
\zy{这是个及物动词,一定要加某人}
\c{prophecy} \yb{"pr6f@si} \n 预言;预言能力;预言书
\cc{prophet} \yb{"pr6fIt} \n 预言家,先知;倡导者,主张者
\jf{-ph- 来自 -graph 写,如 geography 地理}
\c{guarantee} \yb{""g\ae r@n"ti:} \n 保证,担保;保证人,保证书;抵押品 \vt 保证,担保

\ti{The body could not \underline{reconcile} himself to the failure, he did not believe that was his lot.
\choice{submit}{commit}{transmit}{permit}}
\zy{-\cg{mit} 动词结尾}
\c{reconcile} \yb{"rek@nsaIl} \vt 使和好,使和解;调停,排解(争端等);〔宗〕使(场所等)洁净;〔船〕使(木板)接缝平滑
\c{console} \yb{k@n"s@Ul} \vt 安慰,慰问 \n (机器的)操纵台,仪表板
\jf{\cg{sol}- 太阳,来自意大利语,如 solar;独自,来自 solo}
\cc{solar} \yb{"s@Ul@} \a 太阳的,日光的;利用太阳能的;根据太阳决定或测定的 \n 日光浴室 \dy{\ty system | \ty power 太阳能}
\cc{inconsolable} \yb{""Ink@n"s@Ul@bl} \a 没法安慰的,极悲痛的,伤心欲绝的
\cc{disconsolate} \yb{dIs"k6ns@l@t} \a 孤独的,郁郁不乐的;怅怅不乐;惆怅
\c{solicit} \yb{s@"lIsIt} \v 恳求;征求;提起;(指娼妇)拉客
\jf{\cg{sol}- 单独,-cit 看作 city,一个人在城市奋斗难免要求人}
\cc{solicitude} \yb{s@"lIsItju:d} \n 关心,挂念,渴望
\cc{conciliate} \yb{k@n"sIlieIt} \vt 使(某人)息怒或友好,安抚,劝慰 \v (使)意见一致,调节 \n 安抚者,劝慰者 \a 意图或可能抚慰或调解的
\c{submit} \yb{s@b"mIt} \vi 顺从,服从;甘受,忍受 \vt 使服从,使顺从;提交,呈送;〔法〕主张,建议
\c{commit} \yb{k@"mIt} \vt 犯罪,做错事;把…托付给;保证(做某事、遵守协议或遵从安排等);承诺,使…承担义务 \dy{\ty a crime 犯罪 | \ty suicide 自杀 | She committed a suicide because of her mental disorder. 她因精神错乱自杀。}
\cc{mental} \yb{"mentl} \a 内心的,精神的,思想的,心理的;智慧的,智〔脑〕力的;〔口〕精神病的,意志薄弱的,愚笨的 \n 精神病患者 \dy{Mentos 曼妥思}
\cc{criminal} \yb{"krImInl} \n 罪犯,犯人 \a 刑事的;犯罪的;可耻的
\cc{discrimination} \yb{dI""skrImI"neISn} \n 歧视;辨别,区别;辨别力,识别力;不公平的待遇 \dy{sexual \ty | racial \ty}
\jf{\cg{dis}- 不同,-crim- 犯罪,同样的罪行有不同的对待即歧视,常见阅读与翻译}
\cc{commitment} \yb{k@"mItm@nt} \n 承诺,许诺;委任,委托;致力,献身;承担义务
\c{transmit} \yb{tr\ae ns"mIt} \vt 传输;传送,传递;发射;传染 \vi 发送信号
\jf{\cg{trans}- 穿过}
\cc{transfer} \yb{tr\ae ns"f3:} \vt 使转移;使调动;转让(权利等);让与 \vi 转让;转学;转乘;转会(尤指职业足球队) \n 转移;调动;换乘;(运动员)转会 \dy{\ty station 换乘站}
\c{transport} \yb{"tr\ae nspO:t} \vt 运送,运输;流放;使欣喜若狂 \n 运输;运输船(机),运输系统;狂喜;流放犯
\jf{-port 即 port 港口}
\cc{export} \yb{"ekspO:t} \v 出口,输出 \vt 传播,输出(思想或活动) \n 出口;输出,出口产品;输出品
\cc{import} \yb{"ImpO:t} \n 进口,进口商品;输入;重要性;意义 \vt 输入,进口;对…有重大关系;意味着 \vi 具重要性
\c{transplant} \yb{tr\ae ns"plA:nt} \vt 移植;移种;移民,迁移;移植(器官、皮肤、头发等) \n (器官、皮肤、头发等的)移植;移植物;移植者 \vi 迁移;移居;经得起移植 \dy{Plants VS Zombies 植物大战僵尸}
\c{heart} \yb{hA:t} \n 心,心脏;感情;要点;胸部 \vt 鼓励;激励 \vi 结心
\jf{-\cg{card} = -\cg{cord} 心,来自 heart}
\cc{cardiac} \yb{"kA:di\ae k} \a 心脏(病)的;(胃的)贲门的 \n 心脏病患者;强心剂
\cc{cardinal} \yb{"kA:dInl} \n 基数;红衣主教;大红色,深红色;女式斗篷 \a 基本的,最重要的;大红色的,深红色的
\cc{according} \yb{@"kO:dIN} \ad 依照;根据 \v 给予(accord的现在分词);使和谐一致;使符合;使适合 \a 相符的,相应的;和谐的
\cc{cordial} \yb{"kO:di@l} \a 热诚的;诚恳的;兴奋的 \n 补品;兴奋剂;甘露酒
\cc{liver} \yb{"lIv@} \n 肝脏;(食用)肝;深赤褐色;生活者 \a 肝味的;深赤褐色的
\jf{没有 liver 就 live 不了}
\cc{lung} \yb{l2N} \n 肺;呼吸器官;〔医〕辅助呼吸的装置;〔英〕可供呼吸新鲜空气的地方
\c{transit} \yb{"tr\ae nzIt} \n 通过;搬运;转变;运输线 \vt 通过;横越;运送;使(望远镜)水平横轴回转 \vi 通过,经过
\c{transparent} \yb{tr\ae ns"p\ae r@nt} \a 透明的;清澈的;易识破的;显而易见的
\jf{-pare 即 pair 一双,两者之间可毫无障碍地相互穿行即透明的}
\c{compare} \yb{k@m"pe@} \v 比较,对照 \vt 比拟,喻为;〔语〕构成 \vi 相比,匹敌;比较,区别;比拟
\cc{comparable} \yb{"k6mp@r@bl} \a 可比较的;比得上的
\zy{重音在第一个音节}
\cc{comparative} \yb{k@m"p\ae r@tIv} \a 相比而言的;比较(上)的;按比较估计的;相当的,还可以的 \n 〔语〕(形容词或副词的)比较级形式;可匹敌者,可比拟物
\c{compete} \yb{k@m"pi:t} \vi 竞赛;竞争;比得上;参加比赛(或竞赛)
\cc{competition} \yb{""k6mp@"tISn} \n 竞争;比赛;竞争者;〔生〕生存竞争
\cc{competitive} \yb{k@m"pet@tIv} \a 竞争的,比赛的;(价格等)有竞争力的;(指人)好竞争的;〔生化〕抑制酶作用的 \dy{\ty advantage 竞争优势}
\cc{advantage} \yb{@d"vA:ntIdZ} \n 有利条件;益处;优越(性);处于支配地位 \vt 有利于;有益于;促进;使处于有利地位 \vi 得益,获利 \dy{take \ty of 利用}
\cc{competent} \yb{"k6mpIt@nt} \a 有能力的,能胜任的;能干的,称职的;足够的,充足的;有决定权的
\cc{competence} \yb{"k6mpIt@ns} \n 能力;技能;相当的资产
\c{permit} \yb{p@"mIt} \vt 许可,准许;默许,放任;允许,容许 \vi 许可,允许 \yb{"p3ːmIt} \n 许可,准许;许可证,执照
\jf{\cg{per}- 穿过,好比我是门卫看着行人通过}
\cc{perfect} \yb{"p3:fIkt} \a 完美的;正确的;优秀的;极好的 \n 〔语〕完成时 \v 使完善;使完备;使完美

\ti{In some cities of North China, the noise pollution is as \underline{pronounced} as that in Tokyo.
\choice{contemptuous}{contagious}{conspicuous}{contemplated}}
\c{pronounce} \yb{pr@"naUns} \v 发音,读 \vt 宣布,宣称;演讲,讲述 \vi 宣判
\c{announce} \yb{@"naUns} \vi 宣布参加竞选;当播音员 \vt 宣布;述说;声称;预告
\c{pronounced} \yb{pr@"naUnst} \a 明显的,显著的;决然的,断然的;强硬的;被说出来的 \v 发音,读(pronounce的过去式和过去分词);宣布,宣称;断言
\c{contemptuous} \yb{k@n"temptSu@s} \a 蔑视的,鄙视的
\c{contempt} \yb{k@n"tempt} \n 轻视;轻蔑;(对规则、危险等的)藐视;不顾
\jf{-tempt- 即 tempt 诱惑}
\cc{tempt} \yb{tempt} \vt 引诱,怂恿;吸引;冒…的风险;使感兴趣 \vi 有吸引力
\cc{attempt} \yb{@"tempt} \vt 试图;尝试 \n 进攻;尝试,冲击
\cc{temptation} \yb{temp"teISn} \n 诱惑,引诱;诱惑物
\jf{\cg{tag}- = \cg{teg}- = \cg{tan}- = \cg{tam}- 接触,来自 tangent 切线}
\c{tangent} \yb{"t\ae ndZ@nt} \n 〔数〕正切;突然转移话题;突兀的转向;(铁路或道路的)直线区间 \a 〔数〕正切的;相切的;切线的;离题的
\c{contagious} \yb{k@n"teIdZ@s} \a (因接触)传染的;有传染性的;传染病的;有感染力的;会蔓延的
\cc{contaminate} \yb{k@n"t\ae mIneIt} \vt 弄脏,污染;损害,毒害 \vi 玷污,毒害,腐蚀(人的思想或品德)
\c{integrate} \yb{"IntIgreIt} \v 合并;成为一体;加入;融入群体
\jf{\cg{in}- 内,-\cg{teg}- 接触,内部接触上即合并,如两块光洁金属长时间接触会因分子热运动而结合在一起}
\cc{integration} \yb{""IntI"greISn} \n 结合;整合;一体化;(不同肤色、种族、宗教信仰等的人的)混合 \dy{global \ty 全球一体化}
\c{global} \yb{"gl@Ubl} \a 全球的,全球性的,有关全球大局的;全面的,整体的,全局的;球形的,球状的,球面的,球体的;〔计〕全程的
\cc{globalize} \yb{"gl@Ub@laIz} \vt (使)全球化,全世界化
\cc{globalization} \yb{""gl@Ub@laI"zeISn} \n 全球化,全球性
\c{conspicuous} \yb{k@n"spIkju@s} \a 明显的;显而易见的;惹人注意的;显目
\jf{\cg{spect}- = \cg{pect} = spic- 看}
\c{contemplate} \yb{"k6nt@mpleIt} \vt 注视,凝视;盘算,计议;周密考虑 \vi 沉思,深思熟虑
\jf{\cg{con}- 一起,-templ- 按 temple 寺庙来记忆}

\ti{Trivial breaches of regulations we can \underline{pass over}, but more serious ones will have to be investigated.
\choice{exceed}{wither}{overpass}{neglect}}
\zy{ones 代表 breaches}
\c{trivial} \yb{"trIvi@l} \a 琐碎的,无价值的;平常的,平凡的;不重要的;〔生〕种的
\c{breach} \yb{bri:tS} \n 破坏;破裂;缺口;违背 \vt 攻破;破坏,违反
\jf{按 break 记忆}
\c{regulation} \yb{""regju"leISn} \n 管理;控制;规章;规则 \a 规定的,必须穿戴的,必须使用的
\jf{reg- = rig- 统治}
\cc{region} \yb{"ri:dZ@n} \n 地区,地域,地带;行政区,管辖区;(大气,海水等的)层,界,境;(学问等的)范围,领域
\cc{regime} \yb{reI"Zi:m} \n 政治制度,政权,政体;管理,方法;〔医〕养生法;(病人等的)生活规则
\cc{regulate} \yb{"regjuleIt} \vt 调节,调整;校准;控制,管理
\cc{regulation} \yb{""regju"leISn} \n 管理;控制;规章;规则 \a 规定的,必须穿戴的,必须使用的
\cc{regular} \yb{"regj@l@} \a 有规律的,定期的;规则的,整齐的;不变的;合格的 \n 正规军;主力(或正式)队员;常客
\cc{irregular} \yb{I"regj@l@} \a 不规则的,不对称的;无规律的;不合规范的,不合法的;不规则变化的 \n 非正规军军人;不规则物;不合规格的产品
\cc{rigid} \yb{"rIdZId} \a 严格的;僵硬的;(规则、方法等)死板的;刚硬的,顽固的
\c{investigation} \yb{In""vestI"geISn} \n (正式的)调查,研究;侦查;科学研究;学术研究 \dy{Federal Bureau of Investigation 联邦调查局}
\c{excessive} \yb{Ik"sesIv} \a 过度的,极度的;过分的;过多的;过逾
\jf{-\cg{cess} 走,如 excess、exceed、process、proceed}
\c{wither} \yb{"wID@} \vt 使枯萎;使畏缩;使衰弱 \vi 凋谢;衰弱;萎缩
\jf{因为 weather 而 wither}
\c{overpass} \yb{"@Uv@pA:s} \n 立交桥,天桥,高架道路 \v 超过;穿过,渡过,通过,越过;忽视;过去,结束
\c{neglect} \yb{nI"glekt} \vt 疏忽;忽略;遗漏;疏于照顾 \n 玩忽;被忽略的状态;怠慢
\c{ignore} \yb{Ig"nO:} \vt 忽视,不顾;〔法律〕驳回(诉讼)
\cc{ignorant} \yb{"Ign@r@nt} \a 无知的,愚昧的;由无知引起的;无学识的
\cc{ignorance} \yb{"Ign@r@ns} \n 无知,愚昧;蒙

\ti{We were discussing the housing problem when a middle-aged man \underline{cut in} and said, ``There's no point in talking about impossibilities''.
\choice{intersect}{interject}{penetrate}{adulterate}}
\c{housing} \yb{"haUzIN} \n 房屋;供给住宅;掩护;外罩
\c{intersect} \yb{""Int@"sekt} \vt 横断,横切,横穿 \v (指线条、道路等)相交,交叉
\jf{\cg{inter}- 相互,-\cg{sect} 切,来自 section,两条线相互切}
\c{insect} \yb{"Insekt} \n 虫,昆虫;卑鄙的人;微贱的人,小人 \a 昆虫的;卑劣的
\jf{-\cg{sect} 一截一截的,昆虫即节肢动物}
\c{interject} \yb{""Int@"dZekt} \vt (突然)插入,插话;打断
\jf{\cg{inter}- 相互,-\cg{ject} 扔,我向你抛出一句话、你向我抛出一句话}
\c{inject} \yb{In"dZekt} \vt (给…)注射(药物等);(给…)注射(液体);(给…)添加;(给…)投入(资金)
\c{reject} \yb{rI"dZekt} \vt 拒绝;抛弃,扔掉;排斥;吐出或呕吐 \n 被拒绝或被抛弃的人或事物
\c{object} \yb{"6bdZIkt} \n 物体;目标;宾语;客体,对象 \vi 不赞成,反对;抱反感 \vt 提出…作反对的理由
\jf{\cg{ob}- 相反}
\cc{obstacle} \yb{"6bst@kl} \n 障碍(物);障碍物(绊脚石,障碍栅栏)
\jf{-stac- 即 stand,反着站即障碍物}
\c{subject} \yb{"s2bdZIkt, s@b"dZekt} \n 主题,话题;学科,科目;〔哲〕主观;实验对象 \a 须服从…的;(在君主等)统治下的 \v 提供,提出;使…隶属 \dy{\ty to 遭受 / 服从}
\jf{\cg{sub}- 向下,扔出去一个话题,使别人服从}
\c{penetrate} \yb{"penItreIt} \vt 穿透,刺入;渗入;秘密潜入;洞悉,明了 \vi 穿透,穿过;进入,渗入;洞悉
\cc{trait} \yb{treIt} \n 特点,特性;少许
\c{adulterate} \yb{@"d2lt@reIt} \vt (尤指食物)掺假;〔废〕奸污,诱奸 \a 掺杂的;掺假的;不纯的;通奸的 \n 掺假
\cc{adult} \yb{"\ae d2lt} \a 成熟的;(智力、思想、行为)成熟的;成年人的;成年的 \n 成年的人或动物
\cc{adolescent} \yb{""\ae d@"lesnt} \n 青少年 \a 青少年的;青春期的;未成熟的
\cc{adolescence} \yb{""\ae d@"lesns} \n 青春期;青年期
\jf{ado- 即 adult,-les- 即 less,比成年人少即青春期}
\cc{adultery} \yb{@"d2lt@ri} \n 通奸,私通;通奸行为
\end{multicols}
答案:BBACA, BADDC, DACBC,DAADB

\section{中国人民大学 2009 年}
\setUY{人大2009}
题干比较长,选项中单词较复杂。
\begin{multicols}{2}
\ti{International sport should create goodwill between the nations, but in the present organization of the Olympics somehow encourages \hh  patriotism.
\choice{obsolete}{aggressive}{harmonious}{amiable}}
\fy{国际体育应该在国家之间建立良好的信誉,但是在目前的奥运会组织中,某种程度上鼓励了激进的爱国主义。}
\zy{but 出现,表示选项与 goodwill 反义}
\c{goodwill} \yb{""gUd"wIl} \n 友好,亲善;好感,青睐;(企业的)信誉,声誉;商誉
\c{present} \yb{"preznt} \a 现在的;目前的;出席的;〔语法学〕现在时的 \n 现在;礼物 \yb{pri"zent} \v 把…交给;颁发;授予,提出;提交,(以某种方式)展现,显示,表现,使发生;使经历,突然出现;显露;产生,主持播放;主持(节目),上演;公演;推出,正式介绍;引见,正式出席;莅临;出现,(口头或书面)表达,表示,交付;提交 \dy{at \ty 当前}
\c{somehow} \yb{"s2mhaU} \ad 以某种方式,用某种方法;〔非正〕不知怎么地,不知道怎样,不晓得什么缘故;设法,想办法,想个方法;莫明其妙地
\zy{插入语,可忽略}
\c{patriotism} \yb{"peItri@tIz@m} \n 爱国主义;爱国心,爱国精神
\jf{pater- = patr- 父亲,来自 father,ph 与 f 类似通假,法语 père 爸爸}
\cc{capital} \yb{"k\ae pItl} \n 首都;资本;资源;大写字母 \a 极好的;死刑的;资本的;首都的
\cc{capitalism} \yb{"k\ae pIt@lIz@m} \n 资本主义(制度);资本(或财富)的拥有;资本(私人占有和生产盈利)的支配地位
\cc{socialism} \yb{"s@US@lIz@m} \n 社会主义
\c{paternal} \yb{p@"t3:nl} \a 父亲的;父亲(般)的;父系的;父方的
\cc{patriot} \yb{"peItri@t} \n 爱国者,爱国主义者
\zy{美国人将国家比作父亲}
\cc{compatriot} \yb{k@m"p\ae tri@t} \n 同胞;同国人 \a 同国人的;同胞的
\cc{expatriate} \yb{""eks"p\ae tri@t} \n 侨民,移居国外者;被逐出国外者 \a 移居国外的;被逐出国外的 \vi 移居国外,放弃原国籍 \vt 使移居国外,使放弃国籍;(依法或强行)放逐,流放
\c{obsolete} \yb{"6bs@li:t} \a 废弃的;老式的,已过时的 \n 废词;被废弃的事物 \vt 淘汰;废弃
\jf{sol- 太阳、独自,来自意大利语 sole 太阳}
\c{isolate} \yb{"aIs@leIt} \vt 使隔离,使孤立 \vi 隔离,孤立 \n 隔离种群 \a 隔离的,分离的;孤立的
\c{console} \yb{k@n"s@Ul} \vt 安慰,慰问 \n (机器的)操纵台,仪表板
\jf{con- 一起,-sole 孤独,孤独的人一起相互安慰}
% \c{inconsolable} \yb{""Ink@n"s@Ul@bl} \a 没法安慰的,极悲痛的
% \c{disconsolate} \yb{dIs"k6ns@l@t} \a 孤独的,郁郁不乐的;怅怅不乐;惆怅
\cc{desolate} \yb{"des@l@t} \a 无人的;荒凉的;孤独的,凄凉的;荒废的 \vt 使荒无人烟,使荒芜;使凄凉,使孤单
\c{aggressive} \yb{@"gresIv} \a 侵略的,侵犯的,攻势的;(美)有进取心的,积极行动的;有进取心的,有闯劲的;好争斗的,借故生端的,爱打架的,要打架的
\jf{-gress 走}
\cc{extravagant} \yb{Ik"str\ae v@g@nt} \a 过度的,过分的;奢侈的,浪费的;放肆的;大量的
\jf{花钱花出圈了,表示极端浪费}
\cc{intervene} \yb{""Int@"vi:n} \v 出面;介入;阻碍;干扰;插嘴;介于…之间
\cc{interfere} \yb{""Int@"fI@} \vi 干预,干涉;调停,排解;妨碍,打扰
\c{harmonious} \yb{hA:"m@Uni@s} \a 和谐的,融洽的;协调的;音调优美的;悦耳的
\c{harmony} \yb{"hA:m@ni} \n 协调;融洽,一致;和谐;〔音〕和声
\jf{ha- 哈,-mony 钱,哈!钱!}
\c{parsimony} \yb{"pA:sIm@ni} \n 异常俭省;极度节俭;吝啬;小气
\jf{parsi- 怕死,花钱从肋叉子上往下扣}
\c{ceremony} \yb{"ser@m@ni} \n 典礼,仪式;礼仪,礼节;虚礼,客气
\jf{cere 来自罗马神话 Ceres 刻瑞斯,谷物和丰收女神。为祈祷来年有个好收成,需要花钱举行祭祀仪式}
\cc{ceremonious} \yb{""ser@"m@Uni@s} \a 好礼仪的,讲究礼节的,正式的;隆重
\dy{Today we gather here \tyc ly. 今天我们在这里隆重集会。}
\c{amiable} \yb{"eImi@bl} \a 和蔼可亲的;温和的
\jf{ami- 爱,来自法语 amateur 爱}
\cc{amateur} \yb{"\ae m@t@} \n 业余爱好者;外行,生手 \a 业余的,非职业的;外行的

\ti{One can understand others much better by noting the immediate and fleeting reactions of their eyes and \hh to expressed thoughts.
\choice{dilemmas}{countenances}{concessions}{junctions}}
\fy{通过注意到他人的眼神和表情对表达出来的思想的迅速而短暂的反应,人们可以更好地理解他人。}
\zy{选项应与 reaction 同义,由 and 连接}
\c{fleeting} \yb{"fli:tIN} \a 疾驰的,飞逝的;短暂的;稍纵即逝;翩
\c{fleet} \yb{fli:t} \n 舰队;船队;车队;港湾,小河 \a 快速的,敏捷的;转瞬即逝的 \vi 疾驰;飞逝 \vt 使(时间)飞逝;消磨
\jf{来自 fly}
\c{reaction} \yb{ri"\ae kSn} \n 反应;反作用力;反动;保守
\jf{re- 相反,-act 反应,即反馈}
\c{dilemma} \yb{dI"lem@} \n 窘境,困境;进退两难
\jf{di- = bi- = ambi- 二}
\c{bicycle} \yb{"baIsIkl} \n 自行车;脚踏车 \v 骑自行车
\c{divert} \yb{daI"v3:t} \vt 使转移;〔正式〕娱乐;转移注意力
\c{diverse} \yb{daI"v3:s} \a 不同的,多种多样的;变化多的;形形色色的
\cc{diversity} \yb{daI"v3:s@ti} \n 多样化,(人在种族、民族、宗教等方面的)多样性;差异;分歧
\cc{diversify} \yb{daI"v3:sIfaI} \v 使多样化,多样化;进入新的商业领域
\c{ambiguous} \yb{\ae m"bIgju@s} \a 含糊的,不明确的;引起歧义的;有两种或多种意思的;模棱两可
\cc{ambiguity} \yb{""\ae mbI"gju:@ti} \n 含糊;意义不明确;含糊的话,模棱两可的话;可作两种或多种解释
\c{ambition} \yb{\ae m"bISn} \n 抱负;渴望得到的东西;追求的目标;夙愿;野心;雄心;志向;抱负
\jf{俺必胜。ambi- 两,人有两种想法,即有野心}
\c{countenance} \yb{"kaUnt@n@ns} \n 表情;脸,面孔;赞同,支持;鼓励 \vt 表示赞同;嘉奖;宽恕
\cc{contain} \yb{k@n"teIn} \vt 包含,容纳;克制,遏制;牵制;包括或由…构成
\c{concession} \yb{k@n"seSn} \n 让步,迁就;(尤指由政府或雇主给予的)特许权;租借地;承认或允许
\jf{con- 一起,-cess- 走,双方一起走一步,即让步}
\c{junction} \yb{"dZ2NkSn} \n 连接,接合;会合点,接合点;(公路或铁路的)交叉路口;(电缆等的)主结点






\ti{People innately \hh for superiority over their peers although it sometimes takes the form of an exaggerated lust for power.
\choice{strive}{ascertain}{justify}{adhere}}
\fy{人们天生力求比同龄人优越,尽管有时表现为夸张的权力欲。}
\zy{选择和 for 搭配的动词}
\c{innately} \yb{I"neItlI} \ad 天赋地;内在地,固有地
\c{superiority} \yb{su:""pI@ri"6r@ti} \n 优越(性),优等;傲慢
\c{peer} \yb{pI@} \vi 凝视;盯着看;隐退,若隐若现;同等,比得上 \n 同辈,同等的人;贵族;同伴,伙伴 \a 贵族的;(年龄、地位等)同等的;相匹敌的
\jf{能成为 pair 一对儿的叫 peer 同辈人}
\c{form} \yb{fO:m} \n 形状,形式;外形;方式;表格 \vt 形成;构成;组织;塑造 \vi 形成,产生;排队,整队
\cc{transformer} \yb{tr\ae ns"fO:m@} \n 变压器;促使变化的(或人物),改革者
\c{exaggerate} \yb{Ig"z\ae dZ@reIt} \v (使)扩大;(使)增加;夸张
\jf{ag- 做}
\c{agent} \yb{"eIdZ@nt} \n 代理人;代理商;药剂;特工;原动力,动因 \vt 由…作中介;由…代理 \a 代理的
\jf{ag- 做,-ent 人,替我们做事情的人即代理人,}
\c{agency} \yb{"eIdZ@nsi} \n 代理;机构;力量
\c{agenda} \yb{@"dZend@} \n 议事日程;待议诸事项一览表;日常工作事项;议程(agendum 的名词复数)
\c{lust} \yb{l2st} \n 强烈的性欲;肉欲,色情;强烈的欲望;渴望,热烈追求 \vi 贪求;渴望;好色 \dy{\emph{Lust, Caution} 《色戒》}
\c{strive} \yb{straIv} \vi 努力奋斗,力求;斗争,力争
\c{ascertain} \yb{""\ae s@"teIn} \vt 弄清,确定,查明 \dy{\ty whether}
\c{justify} \yb{"dZ2stIfaI} \vt 证明…有理;为…辩护;对…作出解释 \vi 整理版面;证明合法
\c{adhere} \yb{@d"hI@} \vi 黏附;附着;坚持;追随 \vt 使粘附;遵循,坚持;追随,依附 \jf{\ty to}
\jf{ad- 加强,-here 这里}
\c{stress} \yb{stres} \n 强调;重音;压力;重力 \vt 重读;〔机械学〕使承受压力;给…加压力(或应力)
\cc{stressful} \yb{"stresfl} \a 有压力的
\cc{distress} \yb{dI"stres} \n 悲痛;危难,不幸;贫困;〔法]扣押财物 \vt 使痛苦,使忧伤;[法〕扣押(财物);使贫困
\c{strategy} \yb{"str\ae t@dZi} \n 策略,战略;战略学
\cc{strategic} \yb{str@"ti:dZIk} \a 战略(上)的;战略性的;有战略意义的;至关重要的
\c{press} \yb{pres} \vt 压;按;逼迫;紧抱;(向…)拥挤;重压,压平;榨取;坚持;推进;征用;强征……入伍;(举重)推举;(高尔夫)过猛击球 \vi 压;逼迫;重压 \n 媒体,报刊杂志,出版社;记者;报道;拥挤的人群;印刷机;榨汁机;挤压,按
\cc{pressure} \yb{"preS@} \n 压(力);压强;大气压;强迫 \v 强迫;迫使;竭力劝说;密封;使…增压 \dy{blood \ty 血压}
\c{depress} \yb{dI"pres} \vt 压下,压低;使沮丧;使萧条;使跌价
\c{just} \yb{dZ2st} \ad 刚才;仅仅,只是;正好;刚要 \a 公正的,合理的;恰当的;合法的;正确的
\cc{justice} \yb{"dZ2stIs} \n 正义;公正;法律制裁;审判员,法官
\cc{justify} \yb{"dZ2stIfaI} \vt 证明…有理;为…辩护;对…作出解释 \vi 整理版面;证明合法
\c{judge} \yb{dZ2dZ} \v 审判,评判;断定 \vt 估计;评价;(尤指)批评;想,认为 \n 法官;裁判员;评判员;鉴定人
\cc{jury} \yb{"dZU@ri} \n 陪审团;(展览会,竞赛等的)全体评审员;舆论的裁决;评判委员会 \a 应急的;〔航〕(船上)应急用的;暂时的
\cc{prejudice} \yb{"predZudIs} \n 成见,偏见,歧视;侵害;伤害 \vt 使有偏见;不利于,损害
\jf{pre- 之前,-judi- 判断,在事情发生前就判断,即偏见}
\cc{jurisdiction} \yb{""dZU@rIs"dIkSn} \n 司法权;管辖权;管辖范围;权限
\c{adjust} \yb{@"dZ2st} \v (改变…以)适应,调整,校正;调准(望远镜等),对准,校正,校准(机械等);核算(盈亏);〔保〕评定(赔偿要求)
\cc{adjustment} \yb{@"dZ2stm@nt} \n 调节,调整;调节器;调解,调停;(赔偿损失的)清算 \dy{strategic \ty 战略性调整}
\c{adherent} \yb{@d"hI@r@nt} \n 支持者,拥护者 \a 粘着的;〔植〕贴生的;(由于协议、合约等而)发生关系的;〔语〕修饰语的

\ti{Some scientists have suggested that Earth is a kind of, zoo or wildlife \hh for intelligent space beings, like the wilderness areas we have set up on earth to allow animals to develop naturally while we observe them.
\choice{conservation}{maintenance}{storage}{reserve}}
\fy{一些科学家建议,地球是智慧空间生物的一种动物园或野生动植物保护区,例如我们在地球上建立的荒野地区,以使动物在观察时自然生长。}
\c{wildlife} \yb{"waIldlaIf} \n 野生的鸟兽等 \a 野生生物的 \dy{\ty reserve}
\c{nature} \yb{"neItS@} \n 自然;天性;天理;类型
\c{observe} \yb{@b"z3:v} \v 观察;研究 \vt 遵守;观察;庆祝 \vi 注意;说;评述;当观察员
\c{conservation} \yb{""k6ns@"veISn} \n 保存;保护;避免浪费;对自然环境的保护
\jf{-serve 来自 save}
\c{preserve} \yb{prI"z3:v} \vt 保护;保持,保存;腌制食物;防腐处理 \vi 保鲜;保持原状;做蜜饯;禁猎 \n 蜜饯;防护用品;禁猎地;独占的事物(或范围)
\c{conserve} \yb{k@n"s3:v} \vt 保护,保藏,保存;〔化,物〕使守恒;将…做成蜜饯 \n 果酱,蜜饯
\cc{conservative} \yb{k@n"s3:v@tIv} \n 保守的人;(英国)保守党党员,保守党支持者 \a 保守的;(英国)保守党的;(式样等)不时新的
\c{reserve} \yb{rI"z3:v} \n 储备;保护区;保留;替补队员 \vt 储备;保留;预约 \vi 预订 \a 保留的;预备的
\c{deserve} \yb{dI"z3:v} \vt 应受;应得;值得 \vi 应受报答;应得报酬;应得赔偿;应受惩罚
\c{maintenance} \yb{"meInt@n@ns} \n 维持,保持;保养,保管;维护;维修
\c{maintain} \yb{meIn"teIn} \vt 保持;保养;坚持;固执己见



\ti{According to the latest report, consumer confidence \hh a breathtaking 15 points last month, to its lowest level in 9 years.
\choice{soared}{mutated}{plummeted}{fluctuated}}
\fy{根据最新报告,消费者信心上个月骤降了 15 点,跌至 9 年来的最低水平。}
\c{latest} \yb{"leItIst} \n 最新事物;最新消息 \a 最近的;最新的;最现代的
\c{confidence} \yb{"k6nfId@ns} \n 信心;信任;秘密 \a 骗得信任的;欺诈的
\jf{-fid- 相信}
\cc{confident} \yb{"k6nfId@nt} \a 确信的,深信的;有信心的,沉着的;大胆的,过分自信的;厚颜无耻的 \n 知己;心腹朋友
\c{confide} \yb{k@n"faId} \vt 吐露(秘密、心事等);委托,托付 \vi 吐露秘密;信任,信赖
\cc{confidential} \yb{""k6nfI"denSl} \a 秘密的;机密的;表示信任的;亲密的
\cc{confidant} \yb{"k6nfId\ae nt} \n 心腹朋友,知己
\jf{confidant 与 confident 发音相同,前者有 -a- 记为爱,表示知己}
\c{fidelity} \yb{fI"del@ti} \n 忠诚,忠实;逼真;保真度;尽责
\c{breathtaking} \yb{"breTteIkIN} \a 非常激动人心的;惊人的;惊险的;使人透不过气来的
\c{soar} \yb{sO:} \vi 高飞;飞腾;猛增,剧增;高耸,屹立 \n 高飞;高涨;高飞范围;上升高度 \vt 高飞越过;飞升到
\jf{嗖的一声 soar}
\c{mutate} \yb{mju:"teIt} \v (使某物)变化;改变;突变;变异
\cc{mute} \yb{mju:t} \a 缄默的;哑的;无声的;(字母)不发音的 \n 哑巴;(乐器上的)弱音器 \vt 减轻(声音);使…柔和
\c{plummet} \yb{"pl2mIt} \vi 垂直落下;骤然跌落 \n 铅锤;坠子;重压物
\cc{plumb} \yb{pl2m} \vt 使垂直;用测铅测;探索 \a 垂直的 \ad 恰恰,正;垂直地 \vi 做管道工 \n 铅锤,测锤
\c{dump} \yb{d2mp} \vt 倾倒;丢下,卸下;摆脱,扔弃;倾销 \vi 突然跌倒或落下;卸货;转嫁(责任等) \n 垃圾场;仓库;无秩序地累积
\cc{dumpling} \yb{"d2mplIN} \n 饺子;汤团;水果布丁;矮胖的人
\c{fluctuate} \yb{"fl2ktSueIt} \vi 波动;涨落 \vt 使波动;使动摇
\jf{flu- 即 flow}
\cc{flu} \yb{flu:} \n 流行性感冒,流感
\cc{influenza} \yb{""Influ"enz@} \n 〔医〕流行性感冒;〔兽医〕家畜流行性感冒;流感



\ti{Melissa is a computer \hh that destroyed files in computers and frustrated thousands of users around the world.
\choice{genius}{virus}{disease}{bacteria}}
\fy{Melissa 是一种电脑病毒,它会破坏电脑中的文件,让全世界成千上万的用户感到沮丧。}
\c{frustrate} \yb{fr2"streIt} \vt 挫败;阻挠;使受挫折 \a 无益的,无效的
\zy{常用口语词}
\c{destroy} \yb{dI"strOI} \vt 破坏,摧毁;消灭,歼灭(敌人);杀死;使失败
\c{ruin} \yb{"ru:In} \vt 破坏,毁灭;使破产;使没落,使堕落;变成废墟 \n 毁灭,灭亡;(常复数)废墟,遗迹;〔灭亡〕的原因,祸根;损失 \vi 被毁灭;破产;堕落
\c{demolish} \yb{dI"m6lIS} \vt 摧毁,拆毁(建筑物等);毁坏,破坏;推翻;驳倒(论点、理论等)
\cc{demon} \yb{"di:m@n} \n 魔鬼;恶魔;精力过人的人;邪念
\c{genius} \yb{"dZi:ni@s} \n 天才;天赋;天才人物;(特别的)才能
\jf{类比 Guinness 吉尼斯记忆}
\c{talent} \yb{"t\ae l@nt} \n 天资,才能;天才,人才;塔兰特,古代的一种计量单位,可用来记重量或作为货币单位;〔旧,非正式〕性感的人
\cc{talented} \yb{"t\ae l@ntId} \a 有才能的,有才干的;能干的
\c{gift} \yb{gIft} \n 赠品,礼物;天赋;赠送;天资 \vt 赋予;向…赠送;天赋权力(或才能等);授予 \dy{Gift is the gift that god gave us. 天赋是上帝给予我们的礼物。}
\c{genuine} \yb{"dZenjuIn} \a 真正的;坦率的,真诚的;血统纯粹的,纯种的;〔医学〕真性的
\jf{gen- 产生}
\c{gene} \yb{dZi:n} \n 〔生〕基因;遗传因子 \dy{mutated \ty 变异基因}
\c{generate} \yb{"dZen@reIt} \vt 形成,造成;产生物理反应;产生(后代);引起
\cc{generation} \yb{""dZen@"reISn} \n 一代人;代(约30年),时代;生殖;产生 \dy{inspired \ty}
\c{general} \yb{"dZenr@l} \a 大致的;综合的;总的,全体的;普遍的 \n 上将;一般;一般原则;常规 \ty{secretary \ty 秘书长}
\cc{secretary} \yb{"sekr@tri} \n 秘书;干事,书记员;部长,大臣
\c{virus} \yb{"vaIr@s} \n 病毒;病毒性疾病;毒素,毒害;〔计算机科学〕计算机病毒
\jf{vi- 生命、看}
\c{bacteria} \yb{b\ae k"tI@ri@} \n 细菌(bacterium的名词复数)
\cc{bacterium} \yb{b\ae k"tI@rI@m} \n 细菌(复数为bacteria)



\ti{The \hh emphasis on examinations is by far the worst form of competition in schools.
\choice{negligent}{edible}{fabulous}{disproportionate}}
\fy{过分强调考试是目前学校中最糟糕的竞争形式。}
\c{emphasis} \yb{"emf@sIs} \n 强调;着重;(轮廓、图形等的)鲜明;突出,重读 \dy{put \ty on}
\jf{-phase 即 face}
\cc{phase} \yb{feIz} \n 阶段;时期;月相;(月亮的)盈亏 \v 分阶段实行;逐步做;使定相
\c{emphasize} \yb{"emf@saIz} \vt 强调,着重;加强语气;使突出
\jf{en- = em- 使动动词}
\cc{enlarge} \yb{In"lA:dZ} \v 扩大,放大;扩展,扩充;拉长说,详述
\c{form} \yb{fO:m} \n 形状,形式;外形;方式;表格 \vt 形成;构成;组织;塑造 \vi 形成,产生;排队,整队
\cc{formal} \yb{"fO:ml} \a (学校教育或培训)正规的;方式上的;礼仪上的 \n 〔美〕须穿礼服的社交集会;〔口〕夜礼服
\c{competition} \yb{""k6mp@"tISn} \n 竞争;比赛;竞争者;〔生〕生存竞争
\c{negligent} \yb{"neglIdZ@nt} \a 疏忽的;粗心大意的;不留心的;懒散
\cc{neglect} \yb{nI"glekt} \vt 疏忽;忽略;遗漏;疏于照顾 \n 玩忽;被忽略的状态;怠慢
\cc{ignore} \yb{Ig"nO:} \vt 忽视,不顾;〔法律〕驳回(诉讼)
\cc{ignorant} \yb{"Ign@r@nt} \a 无知的,愚昧的;由无知引起的;无学识的
\cc{ignorance} \yb{"Ign@r@ns} \n 无知,愚昧;蒙
\c{edible} \yb{"ed@bl} \a 可以吃的,可食用的 \n 食物
\c{fabulous} \yb{"f\ae bj@l@s} \a 极好的,极妙的;(美貌)惊人的;寓言般的;难以置信的
\cc{fantastic} \yb{f\ae n"t\ae stIk} \a 极好的;很大的;怪诞的;不切实际的
\cc{marvellous} \yb{"mA:v@l@s} \a 不可思议的;惊奇的;极好的;绝妙的
\cc{superb} \yb{su:"p3:b} \a 极好的;华丽的;丰盛的,豪华的;杰出的
\cc{splendid} \yb{"splendId} \a 壮观的,豪华的;极好的或令人满意的;闪亮的;为众人所推崇的
\cc{amazing} \yb{@"meIzIN} \a 令人惊异的 \vt 使大为吃惊,使惊奇(amaze的现在分词);使惊异:感到非常好奇 \n 吃惊;好奇
\cc{magnificent} \yb{m\ae g"nIfIsnt} \a 壮丽的;伟大的,高尚的;华丽的,高贵的;瑰丽的
\c{disproportionate} \yb{""dIspr@"pO:S@n@t} \a 不成比例的;不相称的;不均衡的;打破平衡
\c{portion} \yb{"pO:Sn} \n 一部分;一份遗产(或赠与的财产);嫁妆;分得的财产 \vt 把…分成份额;分配;把…分给(to);命运注定
\c{proportion} \yb{pr@"pO:Sn} \n 部分,份额;比例;匀称;面积,规模


\ti{The boy seemed more \hh to their poverty, after seeing how his grandparents lived.
\choice{reconciled}{consolidated}{deteriorated}{attributed}}
\fy{在看到他的祖父母的生活之后,这个男孩似乎更甘心于他们的贫穷。}
\c{poverty} \yb{"p6v@ti} \n 贫穷;缺乏,不足;贫瘠,不毛;低劣
\c{power} \yb{"paU@} \n 〔机〕动力,功率;力量;政权,权力;强国,大国 \vt 运转;用发动机发动;使…有力量 \vi 靠动力行进;快速行进 \a 权力的;机械能的,电动的;用电力(或动力)发动的
\zy{英语中有很多读音相近、含义相反的单词}
\c{victory} \yb{"vIkt@ri} \n 胜利;克服;成功
\c{victim} \yb{"vIktIm} \n 牺牲者,受害者;自找苦吃的人;受骗者,上当者;为祭祀杀死的动物(或人)
\c{reconcile} \yb{"rek@nsaIl} \vt 使和好,使和解;调停,排解(争端等);将就,妥协 \dy{be \tyc ed to | \ty with}
\zy{考察频率特别高}
\c{console} \yb{k@n"s@Ul} \vt 安慰,慰问 \n (机器的)操纵台,仪表板
\c{conciliate} \yb{k@n"sIlieIt} \vt 使(某人)息怒或友好,安抚,劝慰 \v (使)意见一致,调节 \n 安抚者,劝慰者 \a 意图或可能抚慰或调解的
\c{consolidate} \yb{k@n"s6lIdeIt} \vt 把…合成一体,合并;巩固,加强;统一;合计金额 \vi 统一;合并;联合
\cc{solid} \yb{"s6lId} \a 固体的;实心的;结实的,可靠的;可信赖的 \n 固体;立体图形;立方体
\c{deteriorate} \yb{dI"tI@ri@reIt} \vt 使恶化 \vi 恶化,变坏
\jf{-teri- 来自 terrible}
\c{attribute} \yb{@"trIbju:t} \vt 认为…是;把…归于;把…品质归于某人;认为某事〔物〕属于某人〔物〕 \yb{"\ae trIbju:t} \n 属性;(人或物的)特征;价值;〔语法学〕定语
\jf{词根 tribute}
\c{tribute} \yb{"trIbju:t} \n 致敬;悼念;贡品;体现 \dy{\emph{The Hunger Games} 《饥饿游戏》}
\c{contribute} \yb{k@n"trIbju:t} \v 贡献出;捐赠(款项);投稿(给杂志等);出力
\cc{contribution} \yb{""k6ntrI"bju:Sn} \n 贡献,捐赠,捐助;捐赠,捐助物;投稿,来稿;〔军〕(向占领地人民征收的)军税 \dy{major \ty}
\c{distribute} \yb{dI"strIbju:t} \vt 分配,散布;散发,分发;把…分类;〔电〕配电
\jf{dis- 分开,皇帝把贡品发放到不同的地方}

\ti{During his two-month stay in China, Tom never \hh a chance to practice his Chinese.
\choice{passed on}{passed up}{passed by}{passed out}}
\zy{传递,放弃,路过,昏过去}

\ti{When a person dies, his debts must be paid before his \hh can be distributed.
\choice{paradoxes}{legacies}{platitudes}{analogy}}
\c{debt} \yb{det} \n 债务;负债情况;义务;罪,过失
\c{pay} \yb{peI} \v 付款;偿还;补偿 \vt 给予;支付 \n 工资;薪水;报答
\c{legacy} \yb{"leg@si} \n 遗产;遗赠
\jf{leg- = lig- 法律}
\c{legal} \yb{"li:gl} \a 法律的;合法的;法定的;法律(上)的 \n 法定权利;依法必须登报的声明
\cc{illegal} \yb{I"li:gl} \a 不合法的,违法的;违反规则的 \n 非法移民,非法劳工;间谍
\c{legislation} \yb{""ledZIs"leISn} \n 立法,制定法律;法律,法规
\c{religion} \yb{rI"lIdZ@n} \n 宗教;支配自己生活的大事;教派;心爱的事物
\jf{政教合一}
\cc{religious} \yb{rI"lIdZ@s} \a 虔诚的;笃信宗教的;宗教的;谨慎的 \n 修士,修女,出家人
\c{oblige} \yb{@"blaIdZ} \vt 强制,强迫;使负债务;使感激;施惠于 \vi 施恩惠;帮忙,效劳
\cc{obligation} \yb{""6blI"geISn} \n 义务,责任;证券,契约;债务;恩惠
\c{paradox} \yb{"p\ae r@d6ks} \n 反论,悖论;似是而非的观点;自相矛盾的人或事;〔物〕佯谬
\jf{dox- 与 opinion 有关,para- 旁边,旁边的想法即旁门左道的}
\c{platitude} \yb{"pl\ae tItju:d} \n 平常的话,老生常谈,陈词滥调
\jf{某人说话像 plate 盘子一样平}
% \c{plate} \yb{pleIt} \n 盘子,盆子;金属板;均匀厚度的片状硬物体;〔摄〕底片,感光版 \vt 镀,在…上覆盖金属板;覆盖;电镀;〔印〕给…制铅板
\cc{cliche} \yb{"kli:SeI} \n 陈词滥调
\c{analogy} \yb{@"n\ae l@dZi} \n 类似,相似;比拟,类比;类推
\c{analyze} \yb{"\ae n@laIz} \vt 〔美〕分析;分解;解释;对…进行心理分析


\ti{He claims that advertising today tends to \underline{portray} women in traditional roles such as cooking or taking care of the baby.
\choice{depict}{advocate}{criticize}{analyze}}
\c{advertising} \yb{"\ae dv@taIzIN} \n 广告;做广告,登广告;广告业 \a 广告的;广告业的 \v 公告;为…做广告(advertise的ing形式)
\c{depict} \yb{dI"pIkt} \vt 描绘,描画;描述
\jf{-pict 来自 picture 图画}
\c{portray} \yb{pO:"treI} \v 扮演(角色);描述;描绘;表现
\c{advocate} \yb{"\ae dv@keIt} \vt 提倡;拥护;鼓吹;为…辩护 \n (辩护)律师;提倡者;支持者
\jf{voc- = vok 即 voice}
\c{vociferous} \yb{v@"sIf@r@s} \a 吵吵嚷嚷的;大声叫喊的;叫嚣的
\c{vocabulary} \yb{v@"k\ae bj@l@ri} \n (某一语言的)词汇;(尤指外语教科书中附有释义的)词汇表
\jf{所有能发出声音的综合,即词汇表}
\c{provoke} \yb{pr@"v@Uk} \vt 激起,挑起;煽动;招致;触怒,使愤怒
\jf{pro- 向前,-voke 声音,用声音激发一个人向前冲}
\c{invoke} \yb{In"v@Uk} \vt 乞灵,祈求;提出或授引…以支持或证明;召鬼;借助
\c{revoke} \yb{rI"v@Uk} \vt 撤销,取消;废除 \vi 有牌不跟
\c{criticize} \yb{"krItIsaIz} \v 批评;评论
\c{critic} \yb{"krItIk} \n 批评家;评论员;批评者;挑剔的人
\c{critical} \yb{"krItIkl} \a 批评的,爱挑剔的;关键的;严重的;极重要的;临界的
\c{criticism} \yb{"krItIsIz@m} \n 批评,批判;鉴定,审定,考证,校勘;苛求,〔哲〕批判主义;评论,评论文章
\c{analysis} \yb{@"n\ae l@sIs} \n 分析,分解;梗概,要略;〔数〕解析;验定

\ti{They achieved more than they had ever dreamed, lending a magic to their familystory that no tale or ordinary life could possibly \underline{rival}.
\choice{confirm}{achieve}{match}{exaggerate}}
\c{rival} \yb{"raIvl} \n 对手;竞争者 \vt 与…竞争;比得上某人 \vi 竞争 \a 竞争的
\jf{来自 river,有河就有战争}
\c{achievement} \yb{@"tSi:vm@nt} \n 完成,达到;成就,成绩
\c{tale} \yb{teIl} \n 传说,传言;(尤指充满惊险的)故事;坏话,谣言;〔古〕计算,总计
\jf{来自 tell}
\c{talent} \yb{"t\ae l@nt} \n 天资,才能;天才,人才
\c{ordinary} \yb{"O:dnri} \a 普通的;平常的;一般的;平庸的
\c{confirm} \yb{k@n"f3:m} \vt 〔法〕确认,批准;证实;使有效;使巩固
\cc{firm} \yb{f3:m} \a 坚固的,坚牢的;坚定的,坚决的;严格的;确定的 \v 使坚固;使坚实 \n 商号,商行;公司;企业;工作集体 \ad 坚定地,稳固地 \vt 使坚定,使牢固 \vi 变坚实,变稳固 \dy{a \ty belief}
\cc{form} \yb{fO:m} \n 形状,形式;外形;方式;表格 \vt 形成;构成;组织;塑造 \vi 形成,产生;排队,整队
\c{match} \yb{m\ae tS} \n 比赛;对手;相配的人(或物);火柴 \vt 相同;适应;使较量;使等同于 \v 使相配,使相称

\ti{The most urgent thing is to find a dump for those \underline{toxic} industrial wastes.
\choice{imminent}{recyclable}{smelly}{poisonous}}
\c{toxic} \yb{"t6ksIk} \a 有毒的;中毒的;因中毒引起的 \n 毒物;毒剂
\c{posion} \yb{p6"zaI@n} \n 阴离子,阴向离子
\c{noxious} \yb{"n6kS@s} \a 有害的,有毒的
\jf{nox- 来自 Nox 罗马神话夜神}
\cc{nuisance} \yb{"nju:sns} \n 讨厌的东西(人,行为)麻烦事;非法妨害,损害;麻烦事
\c{pernicious} \yb{p@"nIS@s} \a 很有害的,恶性的,致命的,险恶的;引起巨大伤害的;毁灭性的
\jf{-nic- 依然来自 Nox}
\jf{per- 都与 through 穿过有关}
\cc{obnoxious} \yb{@b"n6kS@s} \a 讨厌的;易受伤害;应受谴责的
\jf{ob- 加强语气,十分有毒}
\c{urgent} \yb{"3:dZ@nt} \a 急迫的;催促的;强求的;极力主张的
\c{merger} \yb{"m3:dZ@} \n (两个公司的)合并;联合体;吸收
\c{imminent} \yb{"ImIn@nt} \a (通常指不愉快的事)即将发生的;迫切的,危急的;逼近的;迫在眉睫
\jf{类比 immediate 记忆。或者,将 -minent 看成 moment 记忆}
\c{recyclable} \yb{""ri:"saIkl@bl} \a 可循环再用的
\c{smelly} \yb{"smeli} \a 发出难闻气味的,有臭味的;〔口〕“smell”的派生
\c{smell} \yb{smel} \n 嗅觉;气味;臭味;发出臭气的人(东西) \v 有(或发出)…气味,闻到,嗅到(气味),闻,嗅(气味),有难闻的气味;散发着臭气,觉察出;感觉到


\ti{British Prime Minister Tony Blair promised the \underline{electorate} that guns would nor be fired without an attempt to win a further U.N. sanction.
\choice{allies}{delegates}{voters}{juries}}





















\ti{The analysis suggests that the tradeoff between our children's college and our own
retirement security is chilling.
\choice{frightening}{promising}{freezing}{revealing}}





















\ti{Their signing of the treaty was regarded as a conspiracy against the British Crown.
\choice{secret plan}{bold attack}{clever design}{joint effort}}





















\ti{Evidence, reference, and footnotes by the thousand testify to a scrupulous researcher
who does considerable justice to a full range of different theoretical and political positions.
\choice{trustworthy}{intelligent}{diligent}{meticulous}}





















\ti{Despite their spartan, isolated lifestyle, there are no stories of women being raped or
wanton violence against civilians in the region.
\choice{intriguing}{exasperating}{demonstrative}{unprovoked}}





















\ti{The gang derived their nickname from their dark clothing and blacked up faces for
nocturnal raids in the forest.
\choice{illegal}{night-time}{brutal}{abusive}}





















\ti{Though sometimes too lazy to work as hard as her sisters, Linda has a more avid
fondness for the limelight.
\choice{mercurial}{gallant}{ardent}{frugal}}

\end{multicols}

% D\. (.*)\n
% }{\1}}\n

% A.
% \\choice{

% [0-9]{1,3}\.
% \\ti{



\printindex
% \printindex[prefix]
% \printindex[suffix]
% \printindex[root]
% \printindex[tubiao]
\end{document}
